\documentclass[12pt, a4paper]{book} %Esboco_Tese_Doutorado_2023-09-20
%\documentclass[tese]{abntex2} %Alternativa de tipo de documento (nesse caso, consultar documentação do pacote abntex2)
\usepackage[utf8]{inputenc} %Codificação
\usepackage[brazilian]{babel} %Idioma do documento e ortografia geral
\usepackage[alf]{abntex2cite}
\usepackage{setspace} %Espaçamento de linhas
\usepackage{graphicx} %imagens
\usepackage[hycap]{caption} %legenda
    %\captionsetup[figure]{labelsep=none} %formato legenda
    \usepackage{chngcntr} %fazer contagem global das figuras
        \counterwithout{figure}{chapter} %fazer contagem global das figuras
\usepackage[colorlinks=true, allcolors=blue]{hyperref} %referência cruzada
    \renewcommand{\thefigure}{\arabic{figure}}
\usepackage{xcolor} %cor da fonte
    \definecolor{textcolor}{RGB}{56,74,103} %cor fonte livro Waldheim
    \definecolor{labelcolor}{RGB}{96,86,78} %cor legenda livro Waldheim
\usepackage{tcolorbox} %caixa de texto
%%%%%%%%%%%%%%%%%%%%%%%%%%%%%%%%%%%%%%%%%%%%%%%%%%%%%%%%%%%%%%%%%%%%%%%%%%
\begin{document}

\frontmatter % Seção de pré-texto (capa, sumário, resumo, etc.)

% Title Page
\title{Traçados Urbanos 

Morfologicamente Adequados:

Diretrizes de Projeto}

\author{Higor Ribeiro da Costa}
\date{Maringá, 2023}

    \maketitle

    \tableofcontents
    \listoffigures %Lista de figuras
    \listoftables %Lista de tabelas

    %\input{resumo} %Arquivo com o resumo da tese

    \mainmatter % Corpo principal da tese

    \onehalfspacing
    

    \part*{}

        \chapter*{Introdução}
        \addcontentsline{toc}{chapter}{Introdução}
        
        Há muito que me pergunto o porquê de nossas cidades serem tão `feias', e penso que isso tenha que ver com seus traçados. Talvez `feio' não seja o melhor termo para descrever esse aspecto da realidade, pois não se trata aqui de um mero juízo estético de minha parte. Porém, ``enquanto tem-se dificuldade para encontrar um parâmetro conjunto para o [termo] `belo', em relação ao [termo] `feio' parece ser menos problemático encontrar um terreno comum'' (Daverio, 2022, \textit{s.p.}, tradução nossa).
            \footnote[1]{\textit{``mentre sul bello si fatica a trovare un parametro congiunto, sul brutto sembra essere meno problematico trovare un terreno comune''} (Daverio, 2022, \textit{s.p.}.} 
        `Se hoje temos tanta tecnologia, por que fazemos casas e prédios, mas, sobretudo, bairros e cidades assim?' Assim `desconjuntados', que `não fazem sentido.' Por que os loteamentos parecem `arranhões de gato' sobre as colinas, com ruas tão íngremes que não permitem a caminhada, dificultam o acesso do transporte, e promovem enxurradas que levam as casas para dentro dos rios? Essa foi uma dúvida que me perseguiu por anos, e, ao começar a entender as causas desse fenômeno, pesquisar uma solução – ou pelo menos uma alternativa – pareceu-me imprescindível.

        $<$FIGURA COM LOTEAMENTO DO TIPO `ARRANHÃO DE GATO'$>$

        É necessário salientar que entre o campo da edilícia e o da cidade há uma lacuna considerável. Quero dizer, quando falo em arquitetura `feia,' quero significar aquela arquitetura que não é orgânica – \textit{i.e.,} cujas partes não são interdependentes – e que não tem uma relação com seu contexto espaço-temporal. Ou seja, uma arquitetura que não conecta passado e futuro em si própria – em termos de soluções técnicas, organizativas, formais, etc. E explicações para esse fenômeno não nos faltam (Caniggia e Maffei, 2008 [1979]; Strappa, 1995).

        Porém, para as cidades já não temos tantas explicações – talvez, precisamente, pela complexidade do tema. O que é uma cidade `feia'? É apenas uma cidade inorgânica? Uma cidade cujo traçado não tem relação com seu contexto – tanto natural como antrópico? E o que fez as cidades se tornarem assim? Ou, em resumo, o que faz com que as novas áreas urbanas tenham, não raro, uma qualidade inferior a antigas áreas urbanas? É certo que vemos as benesses das infraestruturas que não existiam no passado, no entanto, por exemplo, as áreas históricas de cidades antigas fazem os olhos de turistas – ainda que atraídos pelo marketing, mas marketing que soube evidenciar as características positivas de tais áreas. E que características seriam essas? Posso apontar duas, pelo menos. A primeira é a coerência entre as edificações da área, não raro em um \textit{continuum} de fachadas que cria um grande cenário urbano – cenário autêntico. E a segunda é a conformação do traçado que lhes dá suporte, com todos os seus elementos – e isso é o que me interessa.

        \begin{center}
        . . . . .
        \end{center}

        No fim de minha dissertação de mestrado, cheguei à conclusão de que é possível projetar traçados urbanos morfologicamente adequados ao contexto. E explico o que quero dizer com isso. `Contexto' aqui é o conjunto de estruturas naturais e antrópicas de uma área com suas respectivas características. `Morfologicamente adequado' quer dizer aquilo que já é, desde sua concepção, coerente com o formato das estruturas do contexto dado pela realidade. E traçado urbano é a marca das estruturas urbanas que o homem desenvolve.

        No caso das estruturas naturais, o que tomo por mais importante é a orografia, a terra com a sua forma, que é sobre onde se assentam as estruturas que o homem desenvolve, seguida pela hidrografia. Uma área pode ser mais íngreme ou suave, mais ou menos extensa, e seu relevo possui uma hierarquia latente, que pode ser destrinchada por meio de cumeadas, pontos de distribuição, assim como por talvegues e pontos de encontro. E, no caso das estruturas antrópicas, temos parcelamentos rurais precedentes, franjas urbanas com loteamentos, e diretrizes viárias. Com isso, temos ruas e avenidas, lotes urbanos e glebas rurais com seus limites. Cada uma dessas estruturas naturais e antrópicas desenvolve uma relação de interdependência, existencial – pois algumas não podem existir sem outras – e morfológica, por meio de seus formatos poligonais e consequentes angulações – bidimensional ou tridimensionalmente. 

        Ou seja, quando digo que um traçado urbano deve ser `morfologicamente adequado ao contexto', quero implicar que cada um de seus elementos (sobretudo ruas, praças, demais espaços abertos, e os lotes e quarteirões que derivam de sua disposição) deve, na máxima medida possível, seguir, primeiro, os formatos dados pela estruturação natural e, segundo, os formatos dados pelos elementos da estruturação antrópica. E isso se opõe ao \textit{laissez-faire} dos traçados concebidos \textit{a priori} e só depois `adaptados' à realidade, que se impõe forçosamente ao projetista contrariado. Um traçado `adequado' é diferente de um traçado `adaptado'. É a morfogênese planejada contraposta ao automatismo.

        \begin{center}
        . . . . .
        \end{center}

        Durante aquela pesquisa, da qual a presente tese não é senão o desdobramento, desenvolvi o conceito de `\textit{rendimento} urbano' – que afirma que deve existir uma ``coerência intrínseca'' entre o traçado urbano e o contexto natural (Costa e Rego, 2019, p. 7); e projetei um traçado urbano hipotético sobre uma área urbana consolidada, comparando-o com o traçado existente e com a legislação local em vigor. Com isso, verifiquei ser possível projetar traçados urbanos `de qualidade' (Costa, 2020, p. 106). Traçados com bom \textit{rendimento} urbano em termos ambientais, espaciais e econômicos.

        O que fiz na dissertação foi uma simulação baseada na síntese de um novo conceito (o \textit{rendimento} urbano) e em um estudo de caso (a partir do qual foram extraídos parâmetros para a avaliação desse conceito). Eu queria mostrar que um traçado urbano adequado ao sítio tinha lugar no mundo contemporâneo das cidades planejadas \textit{a priori}, posto que, hoje, um processo de desenvolvimento gradual da estrutura urbana, do traçado urbano, parece já não ter lugar – pois o \textit{status quo}, hoje, é o da morfogênese substituída pelo `mecanicismo'. 

        Outrora, as ruas não eram senão a afirmação de percursos pré-existentes, sulcados ao longo do tempo no relevo do território por inúmeras gerações que nos precederam (Caniggia e Maffei, 2008).   Esses percursos, primitivamente utilizados apenas como rotas de passagem, passaram a ser a estrutura de acesso a áreas inicialmente de caça e coleta, posteriormente de cultivo, até chegar à sua partição em propriedades. E, nos locais mais propícios, tais percursos tiveram seus formatos consolidados, consagrados na matéria, por meio das fachadas as edificações que os margeavam. Era a formação do que, no universo lusófono, chamamos de ``rua", com a série de edificações a ela rentes.

        Observando esse processo, não é difícil perceber que eram os saberes tradicionais da consciência espontânea e a acomodação ao legado das gerações anteriores que capitaneavam a formação de ruas – ou melhor, de `percursos edilícios'. E o direito consuetudinário os mantinha com suas características. Hoje, porém, temos leis positivistas que ditam de antemão como um projeto pode ser feito – seja um arruamento, um parcelamento ou um \textit{masterplan}. E é esse projeto que vai moldar a realidade material que constituir-se-á em um sítio. É toda uma outra dinâmica. Assim, naquele momento decidi projetar um novo traçado urbano adequado às exigências da contemporaneidade, porém projetado a partir de um esquema `à antiga'.

        Para projetar esse novo traçado urbano 'à antiga',
            \footnote[2]{Parece contraditório falar em morfogênese – em um processo totalmente espontâneo – e, no entanto, projetar um traçado – ou seja, executar uma atividade apriorística, ainda que esta leve em conta o processo de formação de traçados de morfogênese espontânea. No entanto, o paradgima atual exige o projeto. E, portanto, não posso me eximir dessa realidade e simplesmente deixar a cargo da iniciativa individual a execução de novos traçados urbanos – ainda mais em um momento no qual a consciência espontânea e o imaginário coletivo encontram-se em uma espécie de caos (Caniggia e Maffei, 2008; Carvalho, 2012), dada a miríade de possibilidades de se fazer algo. Mais ainda: dentro da nossa realidade atual existem inúmeros projetistas, gestores, pesquisadores, docentes e alunos que projetam traçados urbanos, e que não vão ceder aos caprichos de um desconhecido.} 
        lancei mão de um estudo de caso, fazendo uma leitura morfológica do traçado original projetado para a cidade de Maringá-PR, reputado como uma solução moderna e adequada às pré-existências do sítio – concomitantemente com o parcelamento rural da Companhia (CTNP/CMNP)
            \footnote[3]{A Companhia que desenvolveu o território rural no qual foi `plantada' a cidade de Maringá era subsidiária da \textit{Paraná Plantantions Ltd.}, de capital britânico, tendo o nome de `Companhia de Terras Norte do Paraná.' No entanto, no ínterim da Segunda Guerra Mundial, com a aquisição da Companhia por investidores brasileiros, ela passou a se chamar `Companhia Melhoramentos Norte do Paraná,' momento no qual Maringá foi pensada e seu traçado encomendado (Rego, 2009).} 
        que encomendou o projeto (Rego, 2009, 2001; Rego \textit{et al.}, 2004; Bonfato, 2008; Beloto, 2015; Meneguetti, 2007; Kohlhepp, 2015; Waibel, 1949). A partir disso, extraí parâmetros de avaliação do \textit{rendimento} urbano para novos traçados urbanos. E, para provar que os era possível utilizar, projetei esse novo traçado urbano sobre uma área da atual cidade de Maringá – outrora parte da área rural parcelada pela Companhia, fora dos limites do plano original da cidade, e com uma `qualidade' inferior a este. Feito isso, desenvolvi um comparativo quantitativo entre os dois traçados urbanos, comparando-os um com o outro, e com a legislação atual (\autoref{fig:comparativo_tracados}), verificando ser possível projetar um novo traçado urbano adequado ao sítio com uma `qualidade' superior ao \textit{modus faciendi} atual e mantendo índices semelhantes, desde que aplicando o conceito de \textit{rendimento}.

        \begin{figure}[h]
            \centering
            \includegraphics[width=1\textwidth]{/Users/Pancratii/GitHub/phd/Sections/Projeto_de_Pesquisa_2023-03-18_Teste/Pictures/comparativo_tracados.png}
            \captionsetup{labelfont=bf}
            \caption{Comparativo entre o traçado existente (à esquerda) e o traçado hipotético projetado (à direita). \textbf{Fonte:} Costa, 2020 (adaptado).}
            \label{fig:comparativo_tracados}
        \end{figure} 

        \begin{center}
        . . . . .
        \end{center}

        A principal evidência que me trouxe até aqui foi a existência de traçados urbanos adequados à topografia do sítio e de traçados feitos à revelia do relevo – estes últimos relacionados a diversos problemas, sendo oriundos daquilo que chamo '\textit{modus faciendi} atual' (Costa \textit{et al.,} 2020). Pude perceber isso em diversas cidades, e não foi diferente com Maringá-PR, meu local de estudo e experimentação até o momento. Nela, o traçado do plano original da cidade – projetado por Jorge de Macedo Vieira – se encaixa na primeira categoria, e o traçado das expansões urbanas – desenvolvido sobre o parcelamento rural da Companhia – na última.  

        O primeiro grupo congrega traçados urbanos orgânicos, com elementos interdependentes que, em geral, não são serializáveis ou intercambiáveis. Tais traçados podem ser oriundos tanto de um desenvolvimento espontâneo como de um processo de planejamento. E, em ambos os casos, o que se vê é um processo de formação ou desenvolvimento projetual mais complexo e elaborado, e, consequentemente, mais prolongado no tempo, adequando-se de modo particular às características físicas do sítio. 

        Já o segundo grupo congrega traçados não-orgânicos e intercambiáveis, oriundos do \textit{modus faciendi} atual. Neles, é possível observar uma lógica `mecanicista' subjacente, na qual prioriza-se um retorno financeiro ligado à venda de lotes, dispostos geralmente em quadras ortogonais.

        O resultado da aplicação indiscriminada de traçados abstratos, de concepção alheia ao contexto no \textit{modus faciendi} atual são ``territórios descontínuos e paisagens contraditórias'' (Strappa, 2018, p. 11, tradução nossa). Loteamentos e loteamentos que `brotam' como fungos a partir das cidades, de suas franjas e conexões.
            \footnote[4]{A expressão `fungos' foi utilizada pelo professor Philippe Daverio (2018), no contexto das cidades italianas, cuja expansão se dá de modo diferente ao que ocorre no Brasil (cf. Indovina, 2009). No entanto, tanto lá como em ultramar, muitas vezes ocorre o surgimento de empreendimentos urbanos (\textit{i.e.,} loteamentos) em locais inusitados, conectados às áreas urbanas consolidadas apenas por meio de uma pequena estrada, assim como os fungos também se reproduzem e espalham `em rede'. Portanto, a analogia permanece válida.} 
        Geram-se, assim, manchas urbanas formadas por traçados desconexos entre si. E o que se percebe é uma tendência à segregação dessas novas áreas urbanas, bem como uma diminuição da mobilidade, com a sobrecarga das poucas vias de acesso – em geral íngremes.

        Além disso, não há uma distinção clara ou uma integração sustentável entre a mancha urbana e o território natural e produtivo. Diversas ruas fazem `incursões' em áreas que, por sua morfologia e características naturais, deveriam ser preservadas. E, desse modo, o que ocorre não é uma integração entre a cidade e o campo, ou o dissolver das estruturas urbanas no território circunstante, mas sim uma espécie de `corrupção' de todos: cidade, campo e natureza.

        \begin{center}
        . . . . . 
        \end{center}

        É verdade que, sobretudo na primeira metade do século XX, houve ideários a tentar sanar essa situação, focando sobretudo na cidade enquanto habitat humano que deveria ser melhorado: estética, logística e funcionalmente. E isso por meio da relação da cidade-campo, da densidade, de determinados arranjos formais do traçado ou da dissolução da cidade tradicional em favor de utopias. Enquanto isso, nos últimos 50 anos, viu-se um emergir de considerações ambientais, não mais da cidade, mas da `paisagem'.
            \footnote[5]{As definições de `paisagem', \textit{`landscape'} e \textit{`paesaggio'} serão destrinchadas no momento oportuno.} 
        E, mais recentemente, podemos ver a ideia de \textit{`landscape'} enquanto \textit{`framework'} das cidades e regiões dentro do chamado \textit{`landscape urbanism'} (Waldheim, 2016). No entanto, em nenhum desses casos é possível identificar um estudo metódico do processo de morfogênese, sobretudo dos assentamentos espontâneos. Isso só se fez visível no arcabouço teórico-metodológico da escola italiana de tipomorfologia urbana – pouco conhecida e divulgada, precisamente por ser uma 'escola' e não um 'ideário' com princípios a aplicar por toda parte. E, ainda assim, a escola italiana trata de leitura e análise urbano-territorial, no entanto sua ênfase projetual se dá na escala das edificações, ou em projetos urbanos (\textit{`masterplan'}). Ou seja, ou oito, ou oitenta.

        Desse modo, não existem diretrizes claras para projetar traçados de maneira coerente e orgânica, sobretudo conforme o conceito de \textit{rendimento} urbano. Ademais, mesmo nos casos dos quais é possível extrair alguma diretriz, percebe-se, como sobredito, que tais casos não são feitos do mesmo modo que os traçados urbanos costumeiramente são feitos – ao menos no Brasil –, \textit{i. e.,} por loteamentos (bidimensionais) – e não à maneira de \textit{masterplan}, em que os edifícios (tridimensionais) – ou ao menos seus volumes – são projetados \textit{a priori} (Maretto, 2018; Maretto, Costa e Rego, 2023); afinal, estou falando de `traçado' (\textit{urban shape}) e não de `forma' (\textit{urban form}). E é diante desse problema que me pergunto: `como projetar um traçado urbano morfologicamente adequado ao contexto?' 

        O que pretendo, portanto, é desenvolver um conjunto de diretrizes para o projeto de traçados urbanos. Traçados esses que devem ser morfologicamente adequados. Diretrizes essas que possam ser aplicadas em diferentes contextos. E meu intuito com isso é gerar uma alternativa ao \textit{modus faciendi} atual. Para isso, eu preciso ampliar o horizonte do conceito de \textit{rendimento} para além do traçado urbano enquanto algo estanque, tornando-o aplicável na dinâmica da realidade. Ou seja, devo relacionar o conceito de \textit{rendimento} urbano com o processo de expansão urbana e com as estruturas naturais e antrópicas que lhe servem de \textit{framework} – nesse caso, expansões extra-urbanas, peri-urbanas e/ou intra-urbanas.
            \footnote[6]{Eu falaria aqui, também, de reestruturação urbana – no sentido caniggiano do termo. No entanto, penso que estabelecer um modo de fazer cidades e territórios adequado ao contexto servirá tanto para `expansões' quanto para `reestruturações' urbanas.} 
        E, a partir disso, desenvolver um artefato (conjunto de diretrizes) que possa ser utilizado por todos aqueles que projetam traçados, e, com isso, produzem cidades e territórios, tais como projetistas, empreendedores, imobiliaristas, legisladores, gestores, pesquisadores, docentes, alunos. Destarte, se falar em mudança de paradigma é de uma presunção deveras arrogante, faz-se mister, porém, falar em `alternativa' ao modo de fazer traçados urbanos hoje em voga.

        \begin{center}
        . . . . .
        \end{center}

        Uma justificativa razoável para a presente empresa é a ausência de estudos prescritivos para traçados urbanos sob a luz do \textit{rendimento} urbano e de sua prerrogativa de coerência entre traçado e contexto. No campo da morfologia urbana, existem estudos de observação e análise de traçados urbanos, como a metodologia \textit{Morpho} (Oliveira e Silva, 2013; Oliveira e Medeiros, 2016), mas que não consideram a topografia; ou estudos que consideram as formas do terreno no território e outros fatores ligados à sustentabilidade ambiental (Fanta \textit{et al.,} 2022), mas que não adentram no tocante específico dos traçados urbanos e de seus formatos. Ambos de caráter não-prescritivo. Neles, vê-se a ausência de diretrizes prescritivas para o desenho das cidades, sobretudo diretrizes que agreguem \textit{savoir-faire} de diferentes áreas do conhecimento. Até existem manuais, porém, ou eles datam de mais de um século, como é o caso de Unwin (1909) e Sitte (1889), ou, mais recentemente, tratam mais de aspectos relacionados ao `aproveitamento' e infraestruturas de loteamentos, como é o caso de Mascaró (2003); e, em ambas as situações, pouco ou nada se fala do aspecto ambiental do traçado urbano. Mais recentemente, até surgem novos manuais, porém, mais voltados ao desenho urbano na escala da rua do que na escala do traçado: canteiros, mobiliário, distribuição das faixas e arborização são seus objetos, não a morfologia urbana em escala mais ampla. Há diretrizes, e até métodos para a análise que dá suporte a um possível processo de projeto, mas não uma prescrição endereçada diretamente a tal. Há ainda o método de projeto desenvolvido no âmbito dos workshops W.A.M. (Maretto, 2018), e aquele presente no trabalho de Saverio Muratori para o programa habitacional INA-Casa (Maretto, Costa e Rego, 2023) – ambos prescritivos. Porém, em ambos, ainda que havendo alguma consideração pela topografia (sobretudo no exemplo muratoriano), parte-se de uma realidade edificada pré-existente, com tecidos urbanos históricos, ou em áreas novas (ou de intervenção) nas bordas de uma realidade construída já consolidada. E isso sem falar que aqui tratamos de projetos \textit{à la masterplan}, ou seja, de um processo que depende do arquiteto e de um ente que possa levar a cabo toda a empreitada, diferente do processo de formação da cidade, com seus diferentes atores e fontes de receita (Alexander \textit{et al.,} 2013).

        Ou seja, não há diretrizes atuais para o projeto de traçados urbanos partindo da característica mais crua e rudimentar de uma cidade, daquilo que lhe é subjacente, a saber: a forma de sua estruturação natural. Esta, com toda sua complexidade, reflexo de relações orgânicas de interdependência entre orografia e hidrografia – moldadas pelo clima, regime pluviométrico, e tipos de solo – recebe todos os impactos do processo de urbanização. Mais um fator a ser observado em um conjunto de diretrizes – o da sustentabilidade ambiental, vinculada ao formato dos traçados, com a qualidade inerente à sua morfologia. Adaptação dos formatos do traçado urbano, que são subjacentes às formas edificadas e que conferem qualidade ao ambiente construído, disposto sobre um relevo natural. Estes são os pontos fulcrais que justificam um estudo como a tese aqui proposta, pois, por meio de um conjunto prescritivo de diretrizes que visa incrementar um processo já existente de projetação de traçados urbanos, explicita-se ser possível associar \textit{rendimento} urbano e `rendimento’ econômico, mostrando que os diversos interesses dos diferentes atores que atuam na cidade podem convergir para um bem comum, no qual todos saiam ganhando.

        Ademais, se é verdade que consegui projetar um traçado urbano adequado a um determinado contexto, não necessariamente é verdade que uma outra pessoa o conseguirá fazer em outro contexto. Faz-se mister pôr à prova o estratagema que utilizei. E como tornar isso palpável? Por meio de um software ou algoritmo, no qual fosse necessario fornecer apenas algumas instruções, dar um \textit{input}? Bom, nesse caso eu precisaria de uma equipe, e diretrizes já delineadas e organizadas metodicamente, coisa que é inviável no prazo de quatro anos de um doutorado – ao menos para um leigo em linguagem de programação. Então por meio de um método, com um passo-a-passo? Não necessariamente, posto que isso engessaria a aplicação do projeto quando em diferentes contextos e circunstâncias. Bem, nesse caso, por exclusão, cheguei à determinação de que são necessárias diretrizes de projeto, um conjunto de diretrizes para o processo de projetar traçados urbanos morfologicamente adequados ao contexto. 

        %Método
        Para desenvolver tais diretrizes, lanço mão da \textit{Design Science Research} (DSR) como método de pesquisa (Dresch \textit{et al.,} 2013; Lacerda \textit{et al.,} 2015), por ser um método prescritivo que visa a melhoria de processos já existentes. No caso em questão, prescritivas são as diretrizes de projeto, e a realidade ou processo existente a ser melhorado é o processo de projeto de traçados urbanos.

        Fulcrais na DSR são: o desenvolvimento de um `artefato' (aqui, o conjunto de diretrizes), e a avaliação e comunicação dos resultados por e para os \textit{stakeholders} (que, no caso em questão, são profissionais, gestores, pesquisadores, docentes e estudantes que lidam com o processo de projeto de traçados urbanos). Tal método de pesquisa é constituído por fases, sendo elas: (1) consciência do problema; (2) sugestão; (3) desenvolvimento; (4) avaliação; e (5) conclusão. Tais fases se retroalimentam entre si, e seus produtos são: (a) proposta; (b) esboço; (c) artefato; (d) mensuração; e (e) resultados (Vaishnavi e Kuechler, 2021 [2004]).

        [17NOV2023 19:56 UEM] Na primeira etapa desse método, deve-se ``identificar e compreender o problema que [se] deseja estudar e solucionar'', bem como ``definir qual é a \textit{performance} necessária para o sistema em estudo'' – performance essa que será medida posteriormente (nas etapas de desenvolvimento e avaliação) com parâmetros extraídos da revisão de literatura e de um estudo de caso. Na segunda etapa, são sugeridas possíveis soluções para o problema de maneira abdutiva – nesse caso, por meio da definição de três situações para traçados hipotéticos. Na terceira etapa, ocorre o desenvolvimento de um dos artefatos propostos na segunda etapa – ou seja, um conjunto de diretrizes que serão extraídas a partir do projeto desses traçados hipotéticos. As diretrizes que se mostrarem adequadas na solução do problema (relativo ao processo de projeto de traçados morfologicamente adequados) serão avaliadas em uma quarta etapa (Dresch \textit{et al.}, 2015, p. 79; Vaishnavi e Kuechler, 2021, pp. 11-13). E é essa disposição das etapas que se desdobra na formatação dos capítulos desta tese.

        Na presente pesquisa, associo a DSR com estratégias como estudo de caso, modelagem, simulação e grupos focais, de modo a compreender a realidade na qual pretendo intervir, identificar diferentes possibilidades dentro da realidade em questão, desenvolver um arcabouço de soluções, e verificar se tais soluções são aplicáveis em outras realidades.

        Assim, para preencher tal lacuna e alcançar o objetivo acima proposto, na senda da \textit{Design Science Research}, fazem-se necessárias as seguintes etapas, que se refletem no delineamento da pesquisa: 
            \begin{itemize}
                \item revisar conceitos e métodos tanto atinentes ao processo de projeto de traçados urbanos como à morfologia urbana, particulamente àqueles afins ao conceito de \textit{rendimento}; 
                \item buscar soluções semelhantes que já tenham sido aplicadas (como \textit{guidelines, patterns,} métodos, leis, manuais, livros) na classe de problema em questão (\textit{i.e.,} ausência de método de projeto), além de selecionar e esboçar um tipo de solução a ser desenvolvida;
                \item efetuar um estudo de caso sobre Maringá, com levantamento documental e cartográfico, leitura morfológica e revisão de literatura acerca do projeto original, das expansões urbanas, do parcelamento rural e das diretrizes viárias, bem como revisão da legislação atual; %e entrevistas com \textit{stakeholders} para entender as premissas, motivações e o processo de projeto no \textit{modus faciendi} atual; 
                \item estabelecer um protocolo para o desenvolvimento do conjunto de diretrizes; 
                \item desenvolver um conjunto de diretrizes piloto por meio de modelagem e simulação (com traçados urbanos hipotéticos); 
                \item avaliar um conjunto de diretrizes por meio de comparativo com legislação, situação, morfologia e \textit{modus faciendi} atuais (para aferir viabilidade), e por meio de grupos focais (para avaliar aplicabilidade em outros contextos); e, por fim, 
                \item refinar e sintetizar as diretrizes do conjunto.
            \end{itemize}

        \begin{center}
        . . . . .
        \end{center}

            \section*{Estrutura da Tese}

            A presente tese, destarte, é estruturada em três 'partes': a primeira, 'consciência'; a segunda, 'desenvolvimento'; e a terceira, 'avaliação'. Na primeira parte, estão presentes o estado da arte, a análise do contexto em que a tese será aplicada – qual circunstância será melhorada/refinada – e a definição de novos termos e conceitos pertinentes à tese aqui desenvolvida. Na segunda, as diretrizes de projeto são extraídas de projetos piloto de traçados urbanos seguindo um protocolo. E, por sua vez, a terceira parte é uma avaliação das diretrizes de projeto por parte dos \textit{stakeholders} do processo de projeto de traçados urbanos, \textit{i.e.,} profissionais, gestores, docentes e estudantes – ou seja, nessa última parte as diretrizes são testadas por terceiros alheios à feitura desta tese, de modo a avaliar se tais diretrizes (artefato) são realmente aplicáveis à realidade, ao meio para o qual foram desenvolvidas, a saber, escritórios (públicos ou privados), ateliês de projeto e salas de aula; e, em seguida, são sistematizadas.

            A primeira parte é dividida nos seguintes capítulos:
            \begin{enumerate}
                \item O traçado da cidade: no primeiro capítulo é feita a revisão assistemática de literatura
                
                \item Maringá, um novo estudo de caso: no segundo capítulo é feito um estudo de caso sobre Maringá, com levantamento documental e cartográfico, leitura morfológica e revisão de literatura acerca do projeto original, das expansões urbanas, do parcelamento rural e das diretrizes viárias, bem como revisão da legislação atual.
                \item A escolha de uma solução: revisão sistemática de literatura para identificar 
            \end{enumerate}

            Na segunda parte, encontra-se o capítulo intitulado `o projeto de traçados hipotéticos'
                Parâmetros: quantitativos (número de lotes, área verde por habitante, área loteável, área pública), leis, etc.
                O projeto de traçados hipotéticos: aqui, apresento três pilotos, projetos hipotéticos de traçados urbanos. O primeiro foi desenvolvido durante o mestrado, sozinho, sobre um promontório inteiro, ou seja, uma espécie de \textit{tabula rasa} e um piloto inicial de traçado urbano. O segundo traçado foi desenvolvido durante enquanto orientei um projeto de iniciação científica, considerando algumas pré-existências antrópicas da área estudada, como diretrizes viárias, avenidas e faixas de domínio pré-existentes; portanto, um piloto de desenvolvimento semi-autônomo das diretrizes por parte de \textit{stakeholders} (eu e o aluno orientado), ao mesmo tempo que abarcando áreas intra-urbanas, o que indica a possibilidade de uso das diretrizes em áreas urbanas sub-utilizadas, ou na pesquisa e no estudo comparativo (acadêmico) de possíveis alternativas ao modo como atualmente se projetam traçados urbanos. E o terceiro foi desenvolvido no âmbito desta tese, considerando um processo de expansão urbana cujos módulos são os lotes rurais paulatinamente ocupados por loteamentos estruturados por uma estrutura comum (eventualmente um \textit{framework}). Neste capítulo, assim, são apresentados tais traçados e o passo-a-passo de seu desenvolvimento projetual, o que nos leva ao capítulo seguinte.
                Diretrizes projetuais: Neste capítulo, as diretrizes projetuais extraídas do desenvolvimento projetual dos pilotos são sistematizadas dentro de um esquema que permite diferentes alternativas, de modo a proporcionar diversas soluções projetuais em diferentes contextos, porém, sempre mantendo a mesma qualidade de morfoadequação do traçado ao contexto.
                Comparativo com existente: (número de lotes, área verde por habitante, área loteável, área pública), leis, etc.
                
            Já na terceira parte, encontram-se os seguintes capítulos:
            \begin{itemize}
                \item Grupos focais e \textit{feedback}
                \item Diretrizes projetuais
            \end{itemize}


    
    \part[Consciência]{Consciência}

        \chapter[O traçado da cidade]{Revisão de literatura}

            \begin{itemize} %Livros para leitura (ver quais parâmetros que eles dão como qualidade e que devem ser levados em conta no momento de projetar um traçado urbano)
                \item Unwin
                \item Sitte
                \item Mascaró %(para tratar do modus faciendi atual, de como se ensina ou o que se consulta na hora de projetar traçados)
                \item Del Rio
                \item Kostof
                \item Vitor Oliveira
                \item Caniggia e Maffei
                \item Álvaro Rodriguez
                \item Waldheim (os dois livros)
                \item Kohlhepp
                \item Fanta et al.
                \item Legislação (brasileira e local) %Levar em conta que isso pode variar no caso de outros países
                \item Carmona
                \item Manual de urbanismo (Gislaine)
            \end{itemize}
                
            Unwin por conta da vinculação de Jorge de Macedo Vieira com a sua prática projetual. Sitte porque Unwin é tributário de Camillo Sitte. Mascaró e Del Rio para entender o que é levado em conta no Brasil no momento de projetar traçados. Wo e Ding para definir o conceito de traçado urbano empregado nesta tese. Vitor Oliveira por suas recentes descobertas no campo da morfologia urbana e sua aplicação no projeto de traçados urbanos. Caniggia e Maffei por trazerem à baila o modo como as cidades se estruturam organicamente. Álvaro Rodriguez e Charles Waldheim por conta da evolução do processo de urbanização e da atual relação entre o território e a cidade – e, por consequência, seu traçado. Em seguida, Kohlhepp e Fanta et al., para entender o tipo de parcelamento rural do território de Maringá, sobre o qual novos traçados urbanos vêm sendo projetados. E, por fim, a legislação, de modo a ter um panorama daquilo que pode ser aplicado na prática no local utilizado para estudo.

            \begin{itemize} %O que analisar no estudo de caso de Maringá
                \item A relação entre território (CTNP) e traçado (esqueleto) em Maringá
                \item O traçado do Vieira (parâmetro de qualidade)
                \item As expansões urbanas (o modo como as expansões urbanas são feitas)
            \end{itemize}

            Para entender o traçado da cidade e o que fazer com ele, proponho duas revisões de literatura. A primeira é uma revisão assistemática, com os principais autores relacionados ao projeto de traçados urbanos, de modo a verificar que parâmetros são divisados por esses autores como qualidades e parâmetros balizadores de projeto. Além disso, tal revisão serve para identificar em que medida obras consagradas – no Brasil e no exterior – ao longo do tempo se aproximam ou não do conceito de \textit{rendimento} e da noção de organicidade a ele associada. Bem como da relação entre cidade e território (sobre o qual se expande).

            A segunda é uma revisão sistemática de literatura para identificar manuais e \textit{guidelines} contendo diretrizes de projeto, no Brasil e no exterior, com o intuito de verificar quais desses manuais e \textit{guidelines} empregam parâmetros de projeto que possam ser relacionados ao conceito de \textit{rendimento} e às qualidades identificadas na revisão assistemática de literatura. Desse modo, será possível identificar artefatos semelhantes ao desenvolvido aqui.

        \begin{center}
        . . . . .
        \end{center} 

            %Mascaró, p. 13
            ``Todo sítio tem na topografia suas caracteristicas principais.'' A frase não é minha, mas de Juan Luis Mascaró (2003, p. 13), autor de um dos mais conhecidos manuais de ``Loteamentos Urbanos''.

            \begin{quotation} 
                \textit{Obviamente, nas declividades, na uniformidade, no tamanho dos morros e das bacias e em outros aspectos do relevo estarão os mais fortes condicionantes do traçado urbano.
                Igualmente, cada sitio tem seu ecossistema natural que, em maior ou menor grau, e alterado e agredido quando sobre ele se faz um assentamento urbano. O novo sistema ecologico criado podera ser agradavel ou não, estável ou instável, econômico ou antieconômico, dependendo, em grande parte, do critério com que o urbanista o trata.
                Não se pode dar uma regra geral, mas geralmente os sistemas mais agradáveis são aqueles que contém menores alterações, tornando-se mais econômicos e estáveis no tempo.
                Com os modernos equipamentos de grande capacidade para os movimentos de terra que tanto orguIham os técnicos dessa area tem-se condições técnicas de criar sítios com topografia totalmente artificial. Frequentemente se vê áreas de relevo complexo serem aterradas e desbastadas completamente, para ali ser criado um perfil topográfico mais simples, objetivando facilitar a subdivisão e a posterior edificação das residencias. Mais simples, sim; melhores, não. 
                Os assentamentos humanos que geralmente mais agradam são aqueles que parecem ter se desenvolvido de forma espontânea, aqueles lugarejos que aparecem como encravados na propria natureza. Curiosamente, esse tipo de assentamento que respeita a natureza é mais economico para implantar, porque dispensa os grandes movimentos de terra. Também se torna mais econômico de manter, porque é ecologicamente mais estável. 
                [E] Visto dessa outra perspectiva, evidencia-se que o desenho urbano não pode ser feito resolvendo apenas o problema na planta. Para se obter um bom desenho, deve-se trabalhar em suas três dimensões, levando em consideração que as soluções escolhidas necessitam se adaptar e serem oriundas das condições topográficas. 
                Embora isso seja muito claro, é frequente encontrar nos compêndios de desenho urbano diferentes traçados alternativos, colocados como se fossem de livre escolha, como se nada tivessem a ver com a topografia.}
            \end{quotation}

        \begin{center}
            . . . . .
        \end{center} 

\section{Unwin}
        \subsection*{Of Civic Art as the Expression of Civic Life}
        \subsection*{Of the Individuality of Towns, with a Slight Sketch of the Ancient Art of Town Planning}

        Leitura do mundo antigo

        ``Classify cities and towns (...) with different types of plan which have been evolved in the course of natural growth or have been designed at different periods by human art.'' (p. 16)

        ``We must admire also the treatment of the road junctions, [and the fact that in the streets the] central space or roadway was open to the sky, [and] the side avenues or footways [were] covered in with a terraced roof'' (p. 45)

        Enquanto no mundo clássico, muito se vê de ângulos retos indo de Norte a Sul e de Leste a Oeste (p. 45-46), nas cidades medievais, as irregularidades `àppear to have so much systema and art in them that there must have been much ore of conscious planning and designing in the laying out these towns that we have been accustomed to think. [Due to the fact that probably] the setting out of the buildings was done largely on the ground by the eye, \textit{and not transfered from a paper plan}.'' (p. 52, destaques nossos).

        \textbf{\textit{Place} = Piazza e ≠ Square. Diferenças conceituais.} Ao longo de todo o texto. O que marca muita ênfase na ideia – que tomará todo um capítulo com diversos exemplos para ser explicada.

        ``The Renaissance of classical learning and art, followed by the introduction of classical traditions and feeling into architecture in the sixteenth and seventeenth centuries, influenced the planning of towns.'' (p. 62)

        ``[The] Renaissance brought with it the power and courage to handle town planning on a large scale, and developed what one may call a grand manner in schemes of laying out.'' (p. 69)

        ``[After] the troubles of the Thirty Years' War were over, the foundation of new towns and town districts in Germany became a favourite occupation of princes. (...) [These towns and discricts] are generally laid out on the straight, formal lines typical of Renaissance work.'' (p. 69, citando Der Städtebau). – %Dá para mesclar com Mascaró, mostrando que foi a partir desse momento que começou-se (novamente) a buscar traçados que `vençam' o relevo ao invés de dialogar com ele, chegando aos nossos traçado de `arrasa tudo'.

        ``tendency is to enclose the corners''

        Unwin dá um exemplo do plano de Sir Cristopher Wren para a reconstrução do centro de Londres após o Grande Incêndio (p. 77), comentando que ele ``designed a plan or model of a new city in which de deformities and inconveniences of the old Town were remedied by enlarging the streets and lands, and carrying them as near parallel to the another (...) avoiding (...) all acute angles, by seating all the parochial churches conspicuous and insular, by forming the most public places into large piazzas the centres of eight eays; by uniting the Halls of twelve chief Companies into one regular space annexed to the Guildhall'''', além de dar outras diretrizes projetuais utilizadas em Londres e exemplos do que conhecemos como código de postura e lei de uso e ocupação do solo (p 79), demonstrando a necessidade de \textit{land readjustment} (p. 80)

        Desse modo, há aqui a confirmação de que Unwin usa o plano de Londres como exemplo em sua obra, a partir da observação das soluções empregadas (a nível de postura) e dos arranjos das ruas (a nível morfológico), bem como de sua implementação (a nível logístico e legal). Ele observa o que foi feito no plano, pois é o que ele irá propor mais tarde (p. 80).

        ``Renaissance School of town planners (...) Such town planning as took place was chiefly on the land of individual owners of large estates, and was generally rigid and formal, until the influence of the landscape gardening school began to extend to the planning of streets (...) to produce curved lines, without much regard to the architectural effect of the buildings.'' (p. 84)

        Isso é importante.  Grandes glebas ou land readjustment. Ou seja, nota-se a dificuldade do planejamento urbano com um parcelamento rural precedente.  BUSCAR referências do parcelamento alemão em KOLHEPP (e dos outros tipos de parcelamento no Brasil: como se faz um parcelamento rural em geral? Como os romanos, quadrado? A partir de linhas retas com limites nas estradas? Ou com as estradas passando por dentro? Como isso pode se relacionar com o traçado urbano?

        ``[Towns] have been allowed to gro in a haphazard manner, each individual owner developing his own land on the lines which suited his own interes of fancy. Too often the only consideration has been to find a plan which would give \textit{the maximum number of building sites at the minimum cost.}'' (p. 88, destaques nossos).

        EXATO! É desse jeito ainda. E é por isso que essa tarefa precisa ser capitaneada por um ente cuja preocupação seja o bem-estar dos cidadãos (como um instituto, companhia, etc.) e não apenas o lucro (como naturalmente sói ocorrer com os proprietários de terra que pensam em lotear suas glebas; proprietários que, ainda que tivessem uma boa vontade maior, dispondo de áreas públicas em seus loteamentos, ainda iriam carecer das conexões com o restante da cidade a depender de seus vizinhos de gleba e seus parcelamentos).

        ``beautifying a town must have been at the bottom of the way in which the work was carried out. In America the tradicion of a formal lay-out, usually on a rigid gridiron or checker-board pattern, has hitherto been little disturbed by any other style. Towns once started on this pattern have continued to grow to an enormous extent (...). The inconvenience and monotony of this arrangement are, however, now compeling the Americans to consider new systems. Diagnoal streets are being arranged, and in some cases cut through the existing blocks.'' (p. 90) %percurso de reestruturação

        ``provision of parks to break up the monotony of towns and provide breathing spaces''
        ``arrangement of wide boulevards (...) to link up the parks''
        ``provide walks and drives about the town, passing through belts of park or garden'' (p. 90)

        Philadelphia example: ``The regularity of the plan has been in various parts broken by traks which had been established before the growth of the town reached these points, but has tended to reasserts itself after passing these roads'' (p. 92)

        ``radial symmetrical diagonals into the grid-iron of the street plan'' (p. 92)

        *``acute-angled plots (...) do not lend (...) production (...) of (...) groups of buildings or (...) useful open spaces. (...) [A] regular system of streets [which not regards] the contours of the ground, (...) entails vast expense in levelling, [and] destroys any interesting character (...) of the site.'' (p. 92) – anotação importante, casa do Mascaró

        ``Camillo Sitte, by a careful study of plans of medieval towns, came to the conclusion that these were designed on lines which not only provided completely for the convenience of traffic, but were in accordance with the artistic principles upon which the beauty of towns must depend. (...) [Now,] the Germans are (...) seeking to reproduce these, and to consciously design along the same irregular lines.''' (p. 98) – É basicamente o mesmo trabalho que estou fazendo. Só que tenho mais dois em meu auxílio: Caniggia e Maffei.

        ``the conscious designind of the modern town planner should be carried out on the same irregular lines'' (p. 104)

        Modern German School of town planners: ``The plans are worked out with increasign detail, and very large scale drawingd of the streets and junctions are prepared before the work is executed'' (p. 104)
        ``transition from the geometrical to the modern systems'' (p. 110).

        Nem só de linhas curvas vive o homem: ``While, however, the importance of most of the principles which Camillo Sitte deduced from his study of medieval towns may be as great as the modern German school thinks, it does seem to me that they are in danger of regarding these principles as the only ones of great importance; nor do they appear to realise how far it is possible to comply with these principles in designs based upon more regular lines. Some of the irregularity in their work appears to be introduced for its own sake, and if not aimlessly, at least without adequate reason; the result being that many of their more recent plans lack any sense of framework or largeness of design at all in scale with the area dealt with.'' (p. 112)

        Unwin mostra mais uma vez sua predileção pelo Land Readjustment: ``One point of great interest in the description which we have quoted of Sir Christopher Wren's plan of London may well be again mentioned here, namely, his proposal that the boundaries of all existing properties should be disregarded, and that the individual parcels of land should all be temporarily given into the hands ot public trustees or commissioners so that they might be rearranged and the area divided, each person receiving back, not his own plot exactly, but as nearly as possible the equivalent of it in the shape of a plot of land arranged to suit the new roads and new groupings of buildings proposed. (...) Where land is held in small lots, some such power of rearranging boundaries seems necessary for good planning to be possible; but there is much discussion among town planners in Germany on this point. Camillo Sitte and those who follow him argue that the necessity chiefly arose owing to the particular geometrical type of planning which was in vogue previous to his day, and that a freer type of planning, in which greater consideration could be shown for the existing conditions of the site for existing roadways and property boundaries, would render needless very much of the rearrangement of properties which the geometrical school of town planning found so necessary. It is further argued that the consideration of these existing conditions would lead to a type of plan having in it something of the interest and variety which characterise the towns of the Middle Ages.'' (p. 113)

        E mostra dois exemplos (p. 111), o primeiro ``'in a scheme which it is obvious could not be carried out without an entire rearrangement of the plot boundaries which are shown'' (p. 113), e o segundo ``laid out on lines suggested by Camillo Sitte in such a way that almost the whole of the existing property lines could be utilised without rearrangement'' (p. 114)

        \subsection*{Of Formal and Informal Beauty}

        Duas classes: formal e informal.
        Duas escolas de designers: ``the work of one being based on the conviction that the treatment should be formal and regular in character, while that of the other springs from an equally strong belief that informality is desirable.'' (p. 116)
        Landscape school: ``bases its work on the admitted beauties to be found in landscape scenes. Finding little or nothing of formality in wild nature, it rather rashly assumes that formality in garden work is unnatural, and the less intelligent section easily passes from such a doubtful premise to the even more doubtful conclusion that the avoidance of formality will produce the natural.''
        `Formal' school:

        ***``the forms which we find beautiful in wild nature are the result, so far as we know, of obedience the most perfect to laws the most complex, so much so that we may call the forms inevitable.'' (p. 119) – Essa é uma observação chave em toda a obra de Unwin.
        ``weather and gravity, (...) materials and the chemical reactions (...) the slopes of the hills and valleys, the bend of the river, the curve of the bay, and the forms of the trees'' (p. 119) – Cf. Caniggia e Maffei, 2008; Carlotti, 1995; (e particularmente) McHarg, 1971.

        Formal ou informal, cada um tem sua beleza:``we are not therefore justified in assuming either that there is any beauty in mere informality, or that informality in the work carried out by men is in any sense natural; nor, on the contrary, can we deduce from such a premise either that formality will not produce beauty or that it is in any sense unnatural for man to do his work on formal lines'' (p. 119)

        ``Any attempt to copy nature must be futile, for the conditions of natural growth are so complex as to be quite beyond the power of the gardener to understand or reproduce.'' (p. 119)
        ``[But] when he introduces direct imitation of nature, by seeking to eradicate all traces of the gardener's hand, and particularly when he does this by the studied avoidance of any formality in the lines of his work, he is attempting to do what is so far beyond his power to do'' (p. 120)

        ``Surely the result of highly revering natural beauty will be to convince us that we cannot create it, that we cannot even successfully imitate it.'' (p. 120) – E por isso, por assim dizer, as ruas da zona 2 formam uma parábola e não um conjunto de curvas que segue 'ipsis litteris' as curvas de nível do promontório (o mesmo se aplica à zona 5).

        \textbf{**É importante fazer um} ``careful study of the site and its possibilities, [with] a reverence for existing natural beauties''
        \textbf{Enquanto isso,} ``[formalists need] to remember that his design is subordinate to the site[; his] formalism must be regarded as a method of carrying out definite aims, and not as an end in itself'' (p. 125)

        \textbf{Abranger de tudo do formal e do informal:} ``The subject is, of course, a wide and difficult one; many of the words, such as “natural,” “formal,” &c., used in discussing it have meanings difficult to define. Both schools of designers include men of such distinguished abilities and produce results of such marked beauty that we need not so much to decide between them as to seek for some third course, not exactly a \textit{via media} but perhaps a \textit{via latior} which shall embrace what is valuable in the aims and methods of each;'' (p. 125)

        ``But the building up of a town is not accomplished by the making of such a sketch design. And even were the artist himself given absolute control of every detail of the work, he would find that modern conditions would upset many of his proportions and that the result when realised would fall far short of his mental picture. But if we plan our towns on somewhat formal lines there are effects of simple, orderly dignity which we may with some probability count on, for the conditions which we shall need to impose on the builder in order to realise them will be few and of a character fairly easily understood.'''' (p. 126)

        ``If the designer is to go to work in a right spirit, he must cherish in his heart a love for all natural beauty, and at the same time have always in his mind a clear appreciation of the beauty of the definite design which he seeks to develop.'' (p. 136)
        ``It will help him to realise the importance of incorporating his design with the site and of so arranging his scheme of laying out that it may serve as a means of harmonising his buildings with the surrounding country. It will save him from rashly destroying trees or other existing features which, with care, might be preserved and incorporated in his design.'' (p. 136)

        ***``The designer who approaches his work in this spirit (...) \textit{must}—be left to decide for himself in each case how far the lines of his site must be followed''. (p. 137) – \textbf{Assim como Vieira fez em Maringá.}

        ***``For example, a certain degree of orderly design in the main lines of a town plan undoubtedly helps materially to the easy understanding and following of it, and in a town so planned a stranger would more readily find his way about, more easily grasp the main lines of direction. But the practical advantages of such an ordefly arrangement of the plan do not require exactitude of symmetry, which often could not be attained without considerable sacrifice of convenience or natural beauty. In such cases it would seem foolish to pay heavily for securing a degree of symmetry only appreciable on a paper plan or from the car of a balloon.'' (p. 137) – \textbf{Note-se em Maringá como as vias curvas seguem um esquema lógico de grid. É um grid que se adapta. E isso, no fim das contas, leva a menos mudanças de direção. E esse é um dos fatores que fazem com que as pessoas tanto elogiem Maringá por ser uma cidade "planejada", isto é, que as ruas não são posicionadas por acaso, ou formando ângulos agudos e triângulos, mas, ao contrário, as ruas "sempre seguem direto", ou melhor, "são sempre retas [i. e., contínuas]", ainda que sejam curvas. 
        Logo, é mais importante a CONTINUIDADE dos percursos do que necessariamente seu FORMATO, pois, efetivamente, o que mais irá contar para a sua experiência são as mudanças de direção (e, em relação às curvas, as perspectivas e possíveis surpresas - como os shoppings sempre tentam imitar).
        (without sacrifice of convenience) Exato. E nesse ponto o Vieira foi muito feliz. (it would seem foolish) Pois é. Diferente da cidade medieval, feita a partir do nível do pedestre, do olho humano.}

        ``On sites much overlooked from high ground, roofs and roof lines become matters of the utmost importance. In fact, the beauty or otherwise of towns, seen from a distance, depends very often rnuch more on the roofs than upon any other part of the buildings.'' \textbf{E isso aqui depende não (apenas, ou não necessariamente) de um projeto paisagístico, mas sim da consideração do tipo edificado, de códigos de postura, e, na base de tudo isso, da disposição das vias (posso ter um terraceamento com telhados justapostos lado a lado que tragam à tona as formas de uma colina de maneira suave e horizontal, ou posso ter um arranhão de gato nessa mesma colina, com as casas dispostas uma após a outra numa fileira vertical que faz denotar ladeiras e escoamentos com declividade acentuada - que é o que mais ocorre atualmente em nossos loteamentos).}

        ``Apart from extremes of formalism and informalism, there is room for a wide divergence of individual preference among designers who accept in general the same guiding principles'' (p. 138).



        \subsection*{Of the City Survey}
        \subsection*{Of Boundaries and Approaches}
        \subsection*{Of Centres and Enclosed Places}
        \subsection*{Of the Arrangement of Main Roads, their Treatment and Planting}
        \subsection*{Of Site Planning and Residential Roads}
        \subsection*{Of Plots and the Spacing and Planing of Buildings and Fences}
        \subsection*{Of Buildings, and how the variety of Each must be dominated by the Harmony of the Whole}
        \subsection*{Of Co-operation in Site Planning, and how Common Enjoyment benefits the Individual}
        \subsection*{Of Bulding Bye-laws}

        \begin{center}
            . . . . .
        \end{center} 

            \section{}%conteúdo anterior [reprovado 10 OUT 2023]
            Alguns parágrafos atrás apontei duas características subjacentes às cidades admiradas por sua `beleza', a saber: o formato (ou configuração) do traçado – do qual falarei adiante – e o \textit{continuum} de fachadas, que deriva de duas realidades latentes na edilícia e estão condensadas nos conceitos de \emph{`tipo'} e \emph{`rendimento'}. E me permito aqui fazer uma analogia com o mundo dos automóveis, ironicamente, para ilustrar essas realidades.

            Se observarmos todos os carros produzidos até hoje (salvo exceções que salpicam de tempos em tempos), todos têm: quatro rodas, um motor, um habitáculo. Certo? E isso está associado com a `função' do automóvel, a saber, ser um veículo que leva passageiros de um ponto a outro. E isso o carro tem em comum com seu ancestral, a carroça, e até com a liteira. Por ter quatro rodas, o carro necessita de pára-lamas, por exemplo. Por ter um motor a combustão (por enquanto), o carro precisa de uma abertura para entrada de ar. Por ter um habitáculo, o carro precisa de portas. E para ser guiado, ele precisa de um volante. E para que seu motorista tenha uma boa utilização dele, ele tem um cockpit e uma série de equipamentos. E tudo isso reverbera na \textit{forma}, no \textit{design} do automóvel. Correto? 

            Muito bem, todas essas são características tipológicas de um automóvel relacionadas à sua `função' e à sua `forma', que segue sua função. E quanto mais um carro é diferente dos outros em suas `linhas', menor será o seu \textit{rendimento} em relação ao conjunto, \textit{i.e.,} ao mundo automotivo. E o que ocorre é que, das duas uma: se ele cai no gosto do freguês, ele `puxa' o mundo automotivo em uma nova direção; ou, se isso não acontece, ele cai no esquecimento. Basta compararmos um Citroën DS 1955 com um Reliant Regal 1953, dois carros com baixo \textit{`rendimento'} – no sentido mencionado – e com dois destinos completamente diferentes. Assim, o \textit{tipo} se forma como um patrimônio de soluções que deram certo, por assim dizer, enquanto que o \textit{rendimento} é a submissão de algo novo a esse patrimônio já existente.
	
	        \begin{center}
		    . . . . .
	        \end{center}

            No mundo das edificações ocorre algo semelhante – ou deveria acontecer. Tomemos as casas como exemplo. As casas acumulam características relacionadas à sua função de residência, de moradia, e, mais primitivamente, de recinto, de abrigo do mundo exterior – e isso se vê bem quando observamos nossa arquitetura do dito período colonial. E isso tanto em planta quanto em fachada. Tudo isso era passado ao longo das gerações e apenas a linguagem mudava de tempos em tempos (aquilo que alguns chamam de `estilo'). Pois bem, isso mudou nas últimas décadas com o avanço da técnica. A cada novidade, buscamos novas formas. Com isso perdemos nosso liame com o passado. E o mesmo ocorre com os traçados urbanos. E, dito isso, voltemos à nossa hipotética pergunta inicial: `por que construímos cidades feias, ou melhor, desarmônicas?' 

            Pensando bem, talvez valha pensar em `o que faz uma cidade ser desarmônica' antes de perguntar o `porquê' disso. E, afinal, qual a razão de saber o `porquê' de uma cidade ser desarmônica? É apenas para trazer à luz o problema? Não penso, pois, embora uma tese científica usualmente se ocupe de identificar e estudar fenômenos, eu sou arquiteto, e, como tal, pretendo desenvolver algo propositivo. E, para isso – aí sim – preciso entender (racionalmente) o fenômeno que fere meus olhos (sentidos), a saber, a tal `feiúra' ou `desarmonia' das cidades. 
	
	        Em princípio, o que salta aos olhos é um aspecto estético, de olhar e ver que as coisas `não encaixam', que não formam um conjunto harmonioso entre si. O cineasta italiano Pierpaolo Pasolini (1974) já dizia que a forma da cidade só se manifesta, aparece e se revela quando confrontada com um cenário natural. E que, portanto, o problema das cidades e o problema da conservação da natureza que as rodeia são uma coisa só.
                \footnote[4]{“La forma della città si manifesta, appare, si rivela se confrontata con un fondale naturale. Proprio la forma della città di Orte pare in quanto tale perché sulla cima di questo colle bruno divorato dall’autunno, con questa brunatura davanti, e contro il cielo grigio. Ora, quelle case che ti ho citato prima (...) vengono turbare soprattutto il rapporto della forma della città e la natura. Ora, il problema della città e il problema della salvezza della natura che circonda la città sono un problema unico. Ma sempre si pone il problema di rispettare il confine naturale tra la forma della città e la natura circostante.” PASOLINI, 1974.} 
            Confesso sou tentado a sonhar com um mundo em que as cidades medievais não deixaram sua forma de polos compactos em meio à campanha, em um território bem estruturado ao longo do tempo. Porém, é necessário lidar com a realidade – pois minha ideia romântica ignora todos os problemas do passado e pinta aquele mundo de rosa.  Nesse sentido, é preciso ver o que está subjacente ao que Pasolini percebeu. O que é que está por trás das mudanças ocorridas nas cidades? 

            Em parte, é possível identificar o que houve, se pensarmos que a edilícia mudou: o abandono, ou melhor, o esquecimento do \textit{tipo} ajuda a entender muita coisa, posto que esse é o primeiro aspecto que salta aos olhos quando estamos `dentro' da cidade, inseridos nela. Porém, esse é apenas um aspecto.\footnote[5]{E, em parte, esse é um aspecto `apenas' visual – apenas em parte, pois a noção de \textit{tipo} perpassa da edificação ao território (CANIGGIA E MAFFEI, 2008). } Há ainda um outro aspecto, que é estrutural. E que está relacionado ao traçado urbano. Assim, se podemos deduzir uma solução em relação às edificações – ou ao menos um vislumbre do que fazer (DALLA NEGRA, 2015; IEVA, 2015; SCARDIGNO, 2021), como nos projetos de Saverio Muratori (MARETTO, COSTA E REGO, 2023) –, o mesmo não se dá em relação ao traçado urbano. E aqui faz-se necessário elucidar a diferença entre duas dimensões a que devemos atentar ao analisar essa as cidades. 
	
	        A primeira é a sua '\textit{forma},' aquilo que podemos enxergar diretamente com os olhos e apalpar com mais facilidade na cidade – isto porque a \textit{forma} urbana (\textit{the urban form}) é, por definição, tridimensional, eu diria até `mais material'. A segunda dimensão é o `traçado urbano' (\textit{the urban shape}), o `formato' da \textit{forma,} o contorno que a delineia, o que, por definição, é bidimensional, e, por isso mesmo, `menos material', ou seja, perceptível pelos sentidos apenas de maneira latente e apenas observável por meio de um processo de abstração que envolve a apreensão da \textit{forma} e sua posterior representação. Representação essa mais esquemática e abstrata do que o seria uma representação direta da \textit{forma} – basta pensar na diferença entre uma pintura, como as de Caspar van Wittel (\autoref{fig:popolo}), e um mapa, como o de Giambattista Nolli (\autoref{fig:nolli_popolo}).

            \begin{figure}[h]
	            \centering
	            \includegraphics[width=1\textwidth]{/Users/Pancratii/GitHub/phd/Sections/Projeto_de_Pesquisa_2023-03-18_Teste/Pictures/popolo.jpeg} 
	            \captionsetup{labelfont=bf}
	            \caption{Vista da \textit{Piazza del Popolo}, em Roma, por Caspar Van Wittel (1652–1736). \textbf{Fonte:} Wikimedia Commons / Sotherby's (coleção privada).}
	            \label{fig:popolo}
            \end{figure}   

            Discorramos rapidamente sobre a ideia de \textit{forma} e traçado. Traçado é o particípio do verbo traçar, \textit{i.e.,} desenhar traços, riscar, sendo o ato ou efeito desse mesmo verbo, resultando, portando, em um `conjunto de traços', ou, no nosso caso, no `desenho que representa uma estrutura arquitetônica ou urbanística', o que é equivalente a dizer `planta', `projeto', ou, para usar um termo mais antigo, `traça' (PRIBERAM, TRAÇADO, 2023). Na língua rústica do latim, `tracto' que dizer `traçar sulcos', e `tractus' é a `delimitação por meio de traços; região, lugar, quarteirão' (FARIA, 1962, p. 1010). Penso que não haja melhor definição que essa: `delimitação por meio de traços' – que corrobora com a ideia de trajetória de estrada ou mesmo linha férrea (PRIBERAM, TRAÇADO, 2023). \textit{Forma,} porém, é um termo mais complexo.

            \begin{figure}
	            \centering
	            \includegraphics[width=1\textwidth]{/Users/Pancratii/GitHub/phd/Sections/Projeto_de_Pesquisa_2023-03-18_Teste/Pictures/nolli_popolo.png}
	            \captionsetup{labelfont=bf}
	            \caption{\textit{La nuova topografia di Roma} (detalhe), de Gianbattista Nolli (1701-1756), com a \textit{Piazza del Popolo} ao norte. \textbf{Fonte:} UC Berkeley Library.}
	            \label{fig:nolli_popolo}
            \end{figure} 

            \textit{Forma} é a `configuração das coisas na parte exterior', o que é equivalente a `feitio' e `formato' (PRIMERAM, FORMA, 2023). No latim, a coisa complica um pouco, com \textit{`forma'} querendo dizer `fôrma' (que em português é um `molde sobre o qual ou dentro do qual se coloca alguma substância fluida, que toma o feitio desse molde'), ou `todo objeto feito na fôrma'; pode, de fato, ser entendida como `desenho, modelo, planta', mas também pode ser entendida como `tipo ideal', ou ainda como `conformação, configuração, constituição' (FARIA, 1962, p. 407). E, se formos para o campo da filosofia, aí é que a confusão aumenta, pois temos um `sentido filosófico e particularmente metafísico', um `sentido lógico', outro `epistemológico', um metodológico', e, por fim, ainda um `sentido estético' (MORA, 2001b, p. 1126) – e, portanto, a ideia de \textit{`forma'} escapa-nos neste momento.

            Desse modo, na presente pesquisa, meu foco se direciona ao `traçado urbano' e não à \textit{`forma} urbana'. No entanto, é necessário esclarecer que o termo `traçado urbano', por sua vez, pode ter mais de uma acepção. Se pensarmos que a \textit{`forma'} da cidade é constituída por seu aspecto tridimensional,
                \footnote[6]{Aqui faço minhas as palavras de José Lamas (2010, pp. 41 e 48), ao dizer que a forma urbana corresponde "ao meio urbano como arquitectura, ou seja, um conjunto de objectos arquitectónicos ligados entre si por relações espaciais" e que é "a materialização no espaço da resposta a um contexto preciso".} 
            com suas edificações e demais estruturas, podemos deduzir que o `traçado' é a marca deixada no solo por essa \textit{`forma'}. Logo, o `traçado' da cidade seria o sulco resultante dos limites dos lotes e quarteirões e dos contornos edificados (que, não raro, definem o próprio desenho das vias, por meio da conjunção de fachadas).
                \footnote[7]{Note-se que, diferente do que atualmente é lugar comum, o que definia o que era ou não a rua, seu limite, seu contorno, seu espaço, era precisamente a fachada da edificação, que imprime essa linha ao mesmo tempo imaginária e real no solo, diferente do que ocorre hoje. Hoje, o lugar comum é aquele de que o que define a via é o meio-fio — ou o limite do lote, que não coincide com a implantação da edificação, com a marca que a mesma deixa sobre o solo. E isso é fruto da desvinculação entre edificação e lote, entre os limites da edificação e os limites do lote. Mas, se olharmos para nosso passado urbano, veremos algo bem diferente. Basta olharmos uma pintura de Caspar Van Wittel da \textit{Piazza Navona} em Roma e veremos como a praça já era muitíssimo bem definida (pelas edificações), mesmo sem qualquer indício de calçamento, e muito menos de meio-fio.} 

            No entanto, aqui, pretendo que o termo `traçado urbano' tome uma ênfase particular. Isso não exclui o sentido aludido acima, posto que tratarei de contornos edificados, lotes e quarteirões ao falar de `traçado urbano' ao longo desta tese – porém serei específico quando o fizer. Desse modo, a ênfase que quero dar é a de `traçado urbano' enquanto sinônimo de `espaço público', tanto na definição de Vitor Oliveira (2016)
                \footnote[8]{\textit{`The public spaces system of a city includes (...) the open spaces for movement, which we designate, in a simplified way, as streets, (...) [and] also the open spaces for permanence, which we designate as squares and gardens.'} (OLIVEIRA, 2016, p. 17).} 
            quanto, particularmente, na visão de Huimin Ji e Wowo Ding (2021) – que trazem uma releitura contemporânea de Gianbattista Nolli (\autoref{fig:nolli_popolo}). Assim, o termo `traçado urbano' se relaciona com as vias, praças e áreas públicas de edificações – espaços com acesso franqueado ao público. Percursos, nós e polos abertos e fechados, cobertos e descobertos, percorríveis em um único plano contínuo: o plano do solo.
                \footnote[9]{Podemos chamar esse plano de `térreo', ainda que ele comporte variações de altitude e inclinação; porém a ideia é a de que esse é o "plano-base" da cidade, a partir do qual se desenvolvem as edificações, para cima ou para baixo.} 
            Por vias podemos tomar percursos como ruas, calçadões, \textit{woonerfs}, escadarias e alamedas e espaços abertos de parques urbanos (excluídos os maciços vegetais), levando em conta a apropriação dos elementos contidos nesses espaços, como ocorre com os canteiros (HANNES, 2016);
                \footnote[10]{Canteiros que, atualmente, são considerados algo à parte, algo que não é inerente à via, mas que a define e delimita, posto que, para boa parte das prefeituras, o que determina o limite da via não é o \textit{continuum} das fachadas das edificações, senão o limite dos lotes (muros), as guias de meio-fio e os canteiros: elementos definidores da rua que não mais estão destinados a um uso comum.} 
            bem como estacionamentos descobertos, além de rodovias e ferrovias (posto que servem de passagem para pessoas e suas mercadorias e que possuem uma marca sobre o solo que se relaciona com o desenho do restante da cidade). Por praças, podemos entender tanto a clássica praça, espaço aberto de dimensões superiores à da rua, o largo, o \textit{pocket park}, desde que descobertos – exceção feita aos \textit{`annodamenti'} (dos quais se tratará no momento oportuno), que ficam em uma situação ambígua de área pública de edificação e praça. E as áreas públicas de edificações podem ser exemplificadas pelas naves das igrejas, pelas platéias dos teatros e pelos halls, corredores, pátios e praças cobertas dos  edifícios públicos, galerias comerciais, mercados e \textit{shopping centers}.

            Dito isso, reforço que: em relação às `formas' edificadas, conhecemos os motivos da crise atual  – e podemos mesmo ir mais a fundo, entendendo as dinâmicas inerentes aos materiais, à economia, às tendências ditadas de tempos em tempos pelas publicações e sua relação com o design de outras áreas. Temos ideia de como estabelecer um liame com o passado, inclusive com exemplos projetuais – e, quando não (como no caso do Brasil), temos um método para `ler' as pré-existências edificadas e, a partir delas, deduzir o \textit{tipo} edilício de um território, de modo a ter o balizador para novas edificações. E nisso reside a força da escola italiana de tipomorfologia urbana. Todavia, mais uma vez: se podemos sabemos como obter respostas em relação à \textit{forma} da cidade, o mesmo não se dá em relação ao seu traçado.

            \section{Rendimento}%conteúdo anterior [reprovado 10 OUT 2023]
            
            \textit{Rendimento}, ``grau de coerência [de algo] com o contexto'' (Maffei, 2003, p. 82, tradução nossa).  Até minha dissertação, pude levantar duas acepções para o termo: uma, o \textit{`rendimento} edilício', e outra, o \textit{`rendimento'} territorial. A primeira é uma dialética entre a ação do homem e uma reação do ambiente (antrópico) no qual ele está inserido (Caniggia e Maffei, 2008, p. 52). A segunda é a medida com que um território pode ser utilizado pelo homem (Carlotti, 1995, p. 19). Dessas duas escalas, pude deduzir uma nova acepção, que intitulei \textit{`rendimento} urbano', ou seja, ``a coerência intrínseca entre o traçado da forma urbana e o contexto natural'' (Costa, 2020, p. 52). Assim, tomando esse ponto de partida, faz-se necessário destrinchar alguns conceitos e definições.

            A proposição inicial da minha tese é a de que `é possível projetar traçados urbanos morfologicamente adequados ao contexto'. Mais que isso, tais traçados não são traçados `adaptados', com uma forma concebida \textit{a priori}, que só depois é confrontada com a realidade e então deformada por ela – como o seria um \textit{grid}, uma matriz de linhas ortogonais. Ao contrário, tais traçados devem ser adequados ao contexto desde sua concepção. Ou seja, o contexto vem antes. É ele que direciona o projeto. Ou melhor, o processo de projeto parte da consideração de suas formas, e, desse modo, o contexto dá a \textit{forma} do traçado urbano. Nesse sentido, vale a pena esmiuçar melhor o que quero dizer com termos como `forma', `traçado', `contexto', `morfologicamente adequado', e outros termos relacionados como `território' e `paisagem' – quero dizer, é necessário mostrar suas acepções e de quais autores extraio tais conceitos.

            \section{Traçado urbano}%conteúdo anterior [reprovado 10 OUT 2023]
                \subsection{Forma urbana}
                \subsection{Elementos antrópicos do traçado: percursos, nós e polos}
                \subsection{Estruturas naturais subjacentes ao traçado}
                    \subsubsection*{Definições de paisagem e território}
                Para Giuseppe Strappa (2014, p. 19, tradução nossa), %https://issuu.com/giuseppestrappa/docs/strappa_diadi_mediterranee_2014
                    ``o território é um modo de olhar o mundo (...), [de ler] a forma das coisas (dos solos, dos percursos, dos assentamentos) para compreender sua estrutura, entender suas origens [e] possíveis transformações''. O território, continua, ``é o conjunto inscindível de solo e trabalho do homem que o habita e transforma, [em suma,] é arquitetura.'' O termo `paisagem', portanto, ``é o aspecto reconhecível da sua estrutura, a sua forma [, que contém um conjunto de contribuições].'' Ou seja, a `paisagem' para Strappa é a forma do território. Um não contém o outro, mas um é a manifestação do outro.\footnote[11]{\textit{``In architettura il territorio è un modo di guardare il mondo. Di leggere, cioè, la forma delle cose (dei suoli, dei percorsi, degli insediamenti) per afferrarne la struttura, capirne le origini, le possibili trasformazioni. Il territorio è l'insieme inscindibile di suolo e lavoro dell'uomo che lo abita e lo trasforma, è architettura. Il termine `paesaggio' è l'aspetto riconoscibile della sua struttura, la sua forma che contiene un insieme di contributi''} (Strappa, 2014, p. 19).}

                    Em uma nota de rodapé, Strappa (2019, \textit{s.p.}) escreve ainda: %http://www.giuseppestrappa.it/wp-content/uploads/2019/10/cap.-10-territorio-per-il-corso.pdf http://www.giuseppestrappa.it/?p=8355
                    \begin{quotation}
                        ``\emph{Land-scape} significa em inglês `modelagem da terra', com ênfase no aspecto natural do ambiente cognitivo com o qual o termo está associado. Ele se opõe ao termo italiano \emph{`paesaggio'} (francês \emph{`paysage'}, espanhol \emph{`paisaje'}), associado ao termo \emph{`paese'} e, portanto, ao latim \emph{`pagus'}, que significa vila, reconhecendo, de maneira concisa, uma relação de solidariedade entre a terra e o assentamento humano. Portanto, a paisagem como expressão cultural está ligada ao espaço habitado, à cooperação entre recursos naturais e artificiais, às transformações que interpretam a forma de picos orográficos, vales, planícies e sua capacidade de se tornar um ambiente construído. Em resumo, [paisagem] é o aspecto visível do território, a expressão concisa de sua estrutura'' (tradução nossa).\footnote[12]{\textit{``Land-scape'' means in English ``modelling of the earth'', with an emphasis on the natural aspect of the cognitive environment the term is associated with. It is opposed to the Italian term ``paesaggio'' (French ``paysage'', Spanish ``paisaje'') associated with the term ``paese'' and hence to the Latin ``pagus'' meaning village, acknowledging, in a concise manner, a relationship of solidarity between the land and human settlement.Therefore, the landscape as a cultural expression is linked to the inhabited space, to the cooperation between natural and artificial resources, to the transformations that interpret the form of orographic peaks, valleys, plains and their ability to become a built environment. In short, it is the territory’s visible aspect, the concise expression of its structure''} (Strappa, 2019, \textit{s.p.}).}
                    \end{quotation}

                \subsection{Relações entre traçado urbano, parcelamento rural e áreas não-antropizadas}
                
                Não é possível reconhecer o traçado urbano como um mero objeto autônomo, sem relação com o solo sobre o qual se assenta, solo esse que também é subjacente ao território que circunda esse traçado. Desse modo, o traçado urbano deve ser entendido dentro do \textit{framework} dado pelo território, ambos sustentados por um solo com características naturais peculiares a cada lugar, formando, assim, uma paisagem só.\footnote[13]{Para Strappa (2013, \textit{s.p.}), a noção de território deriva do nexo entre a ideia de contexto natural e transformações antrópicas desse mesmo contexto natural, o que o faz entender a paisagem como aspecto visível de uma estrutura de relações que conecta os diversos graus e escalas do universo construído dentro da noção de organismo. Escreve: \textit{``Il concetto di territorio deriva dal nesso che lega l’idea di suolo naturale a quella delle trasformazioni artificiali operate dall’uomo nel processo di antropizzazione (trasformazione abitativa e produttiva) del suolo stesso. Noi cogliamo questo processo attraverso momentanei stati di equilibrio che  restituiscono un’idea discreta di una sequenza storica che è, invece, flusso continuo di modificazioni e rivolgimenti.
                Per questo non è comprensibile il senso storico-processuale di un organismo urbano o di un sistema di percorrenze, se non si colloca la loro formazione all’interno di un rapporto di necessità con l’insieme delle relazioni instaurate nel tempo e nello spazio entro il proprio intorno territoriale. Questa forma del territorio antropizzato non é che l’aspetto visibile di una struttura di relazioni che lega nella nozione di organismo i diversi gradi scalari del costruito e che indicheremo col termine `paesaggio'.''} Essa definição mesma de `paisagem' serve para indicar o que é o contexto antrópico.}

            \section{\textit{Modus faciendi} atual}
                \subsection{Manuais, guias e leis de parcelamento do solo}
                \subsection{Entrevistas sobre princípios que norteiam o desenho dos traçados}
                \subsection{Rendimento econômico}
            \section{Morfoadequabilidade: uma nova definição} %Ver se não fica melhor antes da seção sobre o modus faciendi atual.

        \chapter[Maringá: um novo estudo de caso]{um novo estudo de caso sobre Maringá}

        O traçado de Maringá é digno de nota não apenas por sua conformação, mas pelo fato de ter conseguido uma configuração ímpar dentro das suas condicionantes, e mantendo um rendimento econômico adequado para a Companhia. É verdade que, olhando a topografia, o traçado de Maringá poderia ter uma configuração totalmente diferente. No entanto, dentro do seu ideário, e baseando-se na ferrovia como ponto de partida, o traçado projetado por Vieira assumiu uma configuração bastante adequada ao contexto natural. A ferrovia foi o pontapé inicial, sendo o elemento balizador do território enquanto framework, território esse desenvolvido a partir das estruturas naturais, ao menos em larga escala, por meio do desenho Waldhufendorf do parcelamento rural. E Vieira, aproveitando esse elemento estruturador (a ferrovia), apoiou o restante do traçado sobre as estruturas naturais circunstantes à ferrovia e em sintonia com as estruturas antrópicas do território – como estradas rurais e carreadores – encaixando o traçado geometrizado em uma topografia orgânica por meio da utilização de curvas e retas.

        \begin{center}
        . . . . .
        \end{center} 

        Como Vieira nunca esteve em Maringá, é de se suspeitar que o material que ele recebeu foi o levantamento topográfico da gleba destinada pela Companhia para a cidade de Maringá. Note-se que era prática da Companhia separar glebas para suas cidades e patrimônios dentro do parcelamento rural do território pertencente à Companhia. E, nesse sentido, apesar de não haver (ao menos não em meu conhecimento até o momento) nenhum mapa em que conste uma `gleba Maringá' ou algo semelhante, é possível notar no Anteprojeto para Maringá a definição das curvas de nível e de tracejados que as delimitam, corroborando com a hipótese(*) levantada. Desse modo, não é de admirar que Vieira tenha inserido vias que continuam para além dos limites do seu projeto, como fios desencapados à espera de continuação, e que, eventualmente, conflitam com o traçado das estradas rurais da Companhia – note-se, por exemplo, a avenida curva a sudoeste do Anteprojeto que não perfaz o \textit{offset} da estrada Cleópatra: aqui a avenida segue de maneira pormenorizada a cumeada do promontório, enquanto a estrada (como se pode ver em outro mapa) apenas se aproxima da cumeada, mas que não se apoia necessariamente nesta, nem nos pontos de relevo mais suave, seguindo uma linha reta desenhada a partir de uma linha de força que conecta o início e o fim do promontório.

        \chapter[A escolha de uma solução]{artefato a desenvolver}

        [09OUT2023 14:40 SCADA] Na busca por uma solução, nem mesmo Waldheim (2016) mostra algo que congregue o desenvolvimento de traçados urbanos com a topografia. Topografia que, segundo ele (p. ?), é aquilo que une arquitetura, urbanismo e paisagem (architecture, urban design and landscape). Ele mostra parques, e o modo como esses parques interagem na dinâmica das cidades, mas não a morfologia das cidades em si.
        O que proponho aqui, por outro lado, é uma 'nova abordagem' em relação à forma dos traçados urbanos, particularmente para as novas áreas urbanas. Novas áreas urbanas essas que não precisam necessariamente estar vinculadas à expansão urbana, mas podem – e devem – estar atreladas, por exemplo, ao preenchimento dos vazios urbanos presentes em nossas cidades e que podem ser remanejados com uma solução morfoadequada que aguegue traçados urbanos e \textit{landscape urbanism}. (Cf. Waterfront Toronto).

    \part[Desenvolvimento]{Desenvolvimento}

        \chapter[Protocolo]{Protocolo de desenvolvimento do artefato}

        \chapter[O projeto de traçados hipotéticos]{Pilotos}
            
            \section{Projeto hipotético sobre \textit{tabula rasa} (\textit{land readjustment}/Dissertação)}
            \section{Projeto hipotético sobre pré-existências (Diretrizes Viárias/PIBIC)}
            \section{Projeto hipotético sobre parcelamento rural}

        \chapter[Diretrizes projetuais]{artefato – versão inicial}

    \part[Avaliação]{Avaliação}

        \chapter[Comparativo com existente]{estudo de caso}

        \chapter[Grupos focais e \textit{feedback}]{}

        \chapter[Diretrizes projetuais]{artefato – versão refinada}

            \section[Comunicação do artefato aos \textit{stakeholders}]{title}

    %\backmatter % Seção de pós-texto (bibliografia, apêndices, anexos, etc.)            

    \part*{}

        \chapter*{Conclusão}
        \addcontentsline{toc}{chapter}{Conclusão}

        \chapter*{Referências}
        \addcontentsline{toc}{chapter}{Referências}

            \bibliography{bibliografia} % Arquivo .bib com as referências bibliográficas
            % Adicione apêndices e anexos conforme necessário

        \chapter*{Anexos}
        \addcontentsline{toc}{chapter}{Anexos}

\end{document}
