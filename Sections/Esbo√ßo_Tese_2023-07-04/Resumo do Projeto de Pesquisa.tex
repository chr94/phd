\documentclass[]{report}
\usepackage[brazil]{babel}
\usepackage{graphicx} %imagens
\usepackage[hycap]{caption} %legenda
\usepackage[colorlinks=true, allcolors=blue]{hyperref}
\usepackage{titlesec}

% Title Page
\title{Morphoadæquabilitas}
\author{Higor Ribeiro da Costa}

\begin{document}
\maketitle

\tableofcontents

\begin{abstract}
\end{abstract}

\setcounter{secnumdepth}{0}

\chapter*{Resumo do Projeto de Pesquisa – Instrumentalização da tese}
    \addcontentsline{toc}{chapter}{Resumo do Projeto de Pesquisa – Instrumentalização da Tese}

    Keywords: \textit{urban form, urban layouts, urban design, urban morphology, morphogenesis}

    É possível projetar traçados urbanos morfologicamente adequados ao contexo. Não 'adaptados' \textit{a posteriori}, mas 'adequados' desde sua gênese, desenhados \textit{a priori} precisa e especificamente para um contexto, não se encaixando em nenhum outro lugar. Em minha dissertação, da qual esta tese não é senão a continuação, pude levantar essa hipótese. Desenvolvendo o conceito de \textit{'rendimento urbano'} – que afirma que deve existir uma “coerência intrínseca entre o traçado da forma urbana e o contexto natural” (Costa e Rego, 2019, p. 7) – e projetando um  traçado urbano hipotético sobre uma área consolidada, comparado com o traçado existente e com a legislação local em vigor, verifiquei ser possível projetar traçados 'de qualidade' de maneira viável (Costa, 2020, p. 106). Traçados com \textit{rendimento urbano}.

    Todavia, se é verdade que consegui projetar um traçado urbano coerente com o relevo, será se outro alcançaria o mesmo resultado? Que diretrizes me guiaram até ali? Elas de fato são 'seguras'? Podem ser aplicadas em 'outros contextos' para obter a mesma 'qualidade'? E o que avaliaria essa 'qualidade'? E essa 'qualidade' – não teria ela relação com outras coisas que não apenas o relevo? Todas essas são perguntas que se põem em minha mente, podendo ser resumidas na seguinte questão: como projetar um traçado urbano morfologicamente adequado ao sítio?

    O que pretendo desenvolver aqui é um método de projeto de traçados urbanos. 'Método' enquanto um passo-a-passo com diretrizes. Diretrizes flexíveis, mas justificadas em suas razões de ser e no grau de sua flexibilidade para alcançar a 'qualidade' que um traçado urbano deve apresentar desde sua concepção, durante o processo de projeto. Para isso, minha base inicial será o conceito de \textit{rendimento urbano}. Todavia, ele será relacionado a temas como sustentabilidade ambiental, planejamento, transportes e economia das cidades, tornando-se mais robusto e abrangente.\footnote[1]{Almejo com isso, outrossim, que o arcabouço teorético da escola italiana de tipomorfologia possa ultrapassar os círculos acadêmicos que tratam de morfologia urbana e desenho das cidades, chegando aos profissionais, gestores e empreendedores responsáveis pelas novas configurações urbanas impressas nos traçados de loteamentos e intervenções urbanas, bem como chegando aos pesquisadores de outras áreas que podem ter uma relação mais afim com a nossa.} Com isso, pretendo consubstanciar um novo conceito que traduza a 'qualidade' mister a um traçado urbano, em suas distintas possibilidades de concepção em diferentes contexto e por diferentes profissionais. E pretendo que o método de projeto aqui proposto sirva para traçados projetados sobre \textit{tabula rasa} ou sobre estruturas pré-existentes, como parcelamentos rurais e áreas urbanas já consolidadas e franjas – almejando, com isso, que ele sirva não apenas para nortear traçados projetados \textit{ex novo}, mas também para orientar intervenções urbanas. 

    Para isso, lanço mão da \textit{Design Science Research} (DSR) como meu método de pesquisa. Sendo um método de pesquisa prescritivo e que visa a melhoria de processos já existentes, a DSR adapta-se bem à minha empreitada, uma vez que pretendo estabelecer diretrizes (prescrições) para o projeto de traçados urbanos (processo existente). Fulcrais na DSR são a existência de um artefato (o método de projeto) e a comunicação dos resultados da pesquisa.



    %Problemas:
    %    – ausência de conceito (morfoadequabilidade)
    %   – ausência de método (método morfogeneticamente adequado de projeto para traçados urbanos – MEPROTU)
    %   – problemas associados à falta de rendimento.

    
    
    
    %\footnote[2]{Quando falo em 'adequação' e 'coerência', penso-as como sinônimo, ou análogos de 'identificação' em sentido filosófico, em que uma coisa é igual à outra. Um traçado urbano deve ser 'idêntico' ao relevo sobre o qual ele se assenta. Em que medida isso se dará? Apenas na medida em que suas formas se adequarem o máximo possível às formas do relevo, identificando-se com elas.  Exemplo clamorosos disso são cidades medievais e favelas. Como reproduzir seu traçado? Mais ainda, como projetar,  em larga escala, um traçado análogo a esses – e sem prejuízo para a implementação de infraestruturas e o acesso do Estado a essas áreas? São pontos importantes a considerar e que serão tratados, primeiro em uma conceituação teórica (que vai além de uma revisão bibliográfica, ou seja, eu vou formar um conceito para poder dar suporte a um método) e depois a verificação de como é possível projetar o que me perguntei.} 
    
   % assim como os assentamentos de origem espontânea. Aparentemente, isso é possível ao se utilizar os parâmetros associados ao conceito de  

\end{document}