\documentclass[]{report}
\usepackage[brazil]{babel}
\usepackage{graphicx} %imagens
\usepackage[hycap]{caption} %legenda
\usepackage[colorlinks=true, allcolors=blue]{hyperref}
\usepackage{titlesec}
\usepackage{tikz}

% Title Page
\title{Morphoadæquabilitas}
\author{Higor Ribeiro da Costa}

\begin{document}
\maketitle

\tableofcontents

\begin{abstract}
\end{abstract}

\setcounter{secnumdepth}{0}

\chapter*{Resumo do Projeto de Pesquisa – Instrumentalização da tese}
    \addcontentsline{toc}{chapter}{Resumo do Projeto de Pesquisa – Instrumentalização da Tese}

    Keywords: \textit{urban form, urban layouts, urban design, urban morphology, morphogenesis}

    É possível projetar traçados urbanos morfologicamente adequados ao contexo. Não 'adaptados' \textit{a posteriori}, mas 'adequados' desde sua gênese, desenhados \textit{a priori} precisa e especificamente para um contexto, não se encaixando em nenhum outro lugar. Em minha dissertação, da qual esta tese não é senão a continuação, pude levantar essa hipótese. Desenvolvendo o conceito de \textit{'rendimento urbano'} – que afirma que deve existir uma “coerência intrínseca entre o traçado da forma urbana e o contexto natural” (Costa e Rego, 2019, p. 7) – e projetando um  traçado urbano hipotético sobre uma área consolidada, comparado com o traçado existente e com a legislação local em vigor, verifiquei ser possível projetar traçados 'de qualidade' de maneira viável (Costa, 2020, p. 106). Traçados com \textit{rendimento urbano}.

    Todavia, se é verdade que consegui projetar um traçado urbano coerente com o relevo, será se outro alcançaria o mesmo resultado? Que diretrizes me guiaram até ali? Elas de fato são 'seguras'? Podem ser aplicadas em 'outros contextos' para obter a mesma 'qualidade'? E o que avaliaria essa 'qualidade'? E essa 'qualidade' – não teria ela relação com outras coisas que não apenas o relevo? Todas essas são perguntas que se põem em minha mente, podendo ser resumidas na seguinte questão: como projetar um traçado urbano morfologicamente adequado ao sítio?

    O que pretendo desenvolver aqui é um método de projeto de traçados urbanos. 'Método' enquanto um passo-a-passo com diretrizes. Diretrizes flexíveis, mas justificadas em suas razões de ser e no grau de sua flexibilidade para alcançar a 'qualidade' que um traçado urbano deve apresentar desde sua concepção, durante o processo de projeto. Para isso, minha base inicial será o conceito de \textit{rendimento urbano}. Todavia, ele será relacionado a temas como sustentabilidade ambiental, caminhabilidade, \textit{thigmotaxis}, \textit{space syntax}, planejamento, transportes e economia das cidades, tornando-se mais robusto e abrangente.\footnote[1]{Almejo com isso, outrossim, que o arcabouço teorético da escola italiana de tipomorfologia possa ultrapassar os círculos acadêmicos que tratam de morfologia urbana e desenho das cidades, chegando aos profissionais, gestores e empreendedores responsáveis pelas novas configurações urbanas impressas nos traçados de loteamentos e intervenções urbanas, bem como chegando aos pesquisadores de outras áreas que podem ter uma relação mais afim com a nossa.} Com isso, pretendo consubstanciar um novo conceito que traduza a 'qualidade' mister a um traçado urbano, em suas distintas possibilidades de concepção em diferentes contexto e por diferentes profissionais. E pretendo que o método de projeto aqui proposto sirva para traçados projetados sobre \textit{tabula rasa} ou sobre estruturas pré-existentes, como parcelamentos rurais e áreas urbanas já consolidadas e franjas – almejando, com isso, que ele sirva não apenas para nortear traçados projetados \textit{ex novo}, mas também para orientar intervenções urbanas. 

    Para isso, lanço mão da \textit{Design Science Research} (DSR) como meu método de pesquisa. Sendo um método de pesquisa prescritivo e que visa a melhoria de processos já existentes, a DSR adapta-se bem à minha empreitada, uma vez que pretendo estabelecer diretrizes (prescrições) para o projeto de traçados urbanos (processo existente). Fulcrais na DSR são a existência de um artefato (o método de projeto) e a comunicação dos resultados da pesquisa – não apenas para a comunidade acadêmica, mas também para os usuários finais do artefato em questão (profissionais, gestores, docentes e alunos). E é por essa razão que ela se aplica à minha pesquisa, em continuidade com a própria dissertação.

    Acerca dos traçados urbanos, existe algo que chamo \textit{'modus faciendi} atual', \textit{i.e.,} a maneira como atualmente se projetam traçados urbanos. Ela pode ser percebida no desenho dos loteamentos, bairros e cidades, reflexos de um processo de projeto intencional, ainda que feito por diferentes atores. Por exemplo: o dono de uma gleba rural busca um profissional para fracioná-la em lotes urbanos agregados em quadras conectadas por ruas. Cada loteamento se assoma ao traçado urbano (cidade) ou proto-urbano (vilarejo) pré-existente, bem como é modulado pelo parcelamento rural – ou área livre – sobre o qual é implantado (Kostof, ; Bertaud, ). Em geral, esses loteamentos só são possíveis graças a uma rede de estradas sobre as quais eles se apoiam. Essas estradas podem  resultar de um processo espontâneo de ocupação do território (ocorrido em regiões mais antigas, nas quais o percurso dessas estradas é apenas reforçado pela implementação de infraestruturas), mas também podem ser resultantes ou de ações pontuais (como com a abertura de uma nova rodovia)ou de um processo planejado de ocupação do território (como um conjunto de novas estradas dentro de um sistema de cidades e distritos). E é aqui que reside o problema. 
    
    Outrora, esse processo de estruturação de uma área para o uso do homem era feito de modo espontâneo, baseado numa cultura desenvolvida ao longo do tempo, e o seu resultado era um traçado que se adequava às características do sítio, particularmente no que tange às suas características físicas. Já no \textit{'modus faciendi} atual' o que se percebe é uma aversão às formas do sítio. As diretri
    Outrora, esse processo de estruturação de uma área para o uso do homem era feito de modo espontâneo, baseado numa cultura desenvolvida ao longo do tempo, e o seu resultado era um traçado que se adequava às características do sítio, particularmente no que tange às suas características físicas. Já no \textit{'modus faciendi} atual' o que se percebe é uma aversão às formas do sítio. Novas ruas são desenhadas apenas em função das quadras e da maior capacidade de lotes vendáveis. Novas avenidas são abertas segundo diretrizes viárias que apenas seguem linhas de força que não se adequam ao contexto – seja com com retas \textit{ad infinitum} ou com curvas ameboides que levam do nada para canto nenhum. O resultado é um conjunto de loteamentos que, brotando como fungos a partir das cidades, de suas franjas e de suas conexões, formam "territórios descontínuos e paisagens contratidórias" (Strappa, 2018, p. 11, tradução nossa).

    Percebe-se como consequência desse \textit{'modus faciendi} atual problemas socioambientais como processos erosivos e segregação socioespacial resultantes de traçados inadequados ao relevo, com ruas que não se conectam numa relação de interdependência, parcelas urbanas que não são dispostas gradativamente em suas dimensões a partir de polaridades urbanas, espaços abertos públicos e equipamentos públicos construídos em áreas residuais, e mobilidade não otimizada graças a todos esses fatores. Todos esses problemas já são estudados. É verdade. Mas não há estudos que façam uma correlação de todos esses fatores. Além disso, idearios diversos já surgiram no intento de melhorar as cidades, do ponto de vista estético, logístico e funcional, todavia nenhum deles tratou da cidade de maneira 'morfológica'. Não há métodos para projetar traçados urbanos de maneira coerente e orgânica, conforme os parâmetros do \textit{rendimento urbano}. Ao menos não considerando a maneira como se pratica costumeiramente, com loteamentos (bidimensionais) – e não à maneira de \textit{masterplan}, em que os edifícios (tridimensionais) são projetados \textit{a priori} (Maretto, 2018; Maretto, Costa e Rego, 2023); afinal, estou falando de 'traçado' (\textit{urban shape}) e não de 'forma' (\textit{urban form}). %Seria isso o 'ambiente externo' da minha DSR? (Talvez parte do ambiente externo).

    Diante disso, a pesquisa é delineada a partir das fases e etapas de uma DSR (Lacerda \textit{et al.}, 2013; Takeda \textit{et al.}, 1990; Santos, 2018). E na primeira fase, de compreensão do problema, pretendo lançar mão de: revisão bibliográfica sistemática (RBS) sobre \textit{'urban layouts design methods'} em bases de dados acadêmico-científicas de modo a encontrar possíveis artefatos já utilizados para o problema aqui apresentado; revisão bibliográfica assistemática (RBA) das contribuições mais recentes da escola italiana de tipomorfologia (visando encontrar novos conceitos e possíveis métodos de projeto dentro da ideias de 'cidade como organismo' e de \textit{rendimento}) e dos temas da sustentabilidade ambiental, caminhabilidade, \textit{thigmotaxis}, \textit{space syntax}, planejamento, transportes e economia das cidades, no intuito de compreender como tais ideias podem ser aglutinadas em um único conceito, além da legislação pertinente ao processo de loteamento – tudo isso levando também à definição de parâmetros de avaliação desse método';  entrevistas a profissionais e gestores, de modo a compreender melhor o \textit{'modus faciendi} atual' da maneira como ele funciona na prática; e um estudo de caso no município de Maringá, compreendendo o plano original (visto na dissertação), as áreas de expansão urbana da cidade e o território circunstante parcelado pela Companhia de Terras Norte do Paraná (CTNP) – constituído por uma leitura morfológica associada à revisão bibliográfica dos princípios norteadores desse processo de planejamento e projeto.
    Diante disso, a pesquisa é delineada a partir das fases e etapas de uma DSR (Lacerda \textit{et al.}, 2013; Takeda \textit{et al.}, 1990; Santos, 2018). E na primeira fase, de compreensão do problema, pretendo lançar mão de: revisão bibliográfica sistemática (RBS) sobre \textit{'urban layouts design methods'} em bases de dados acadêmico-científicas de modo a encontrar possíveis artefatos já utilizados para o problema aqui apresentado; revisão bibliográfica assistemática (RBA) das contribuições mais recentes da escola italiana de tipomorfologia (visando encontrar novos conceitos e possíveis métodos de projeto dentro da ideias de 'cidade como organismo' e de \textit{rendimento}) e dos temas da sustentabilidade ambiental, caminhabilidade, \textit{thigmotaxis}, \textit{space syntax}, planejamento, transportes e economia das cidades, no intuito de compreender como tais ideias podem ser aglutinadas em um único conceito, além da legislação pertinente ao processo de loteamento – tudo isso levando também à definição de parâmetros de avaliação do artefato proposto;  entrevistas a profissionais e gestores, de modo a compreender melhor o \textit{'modus faciendi} atual' da maneira como ele funciona na prática; e um estudo de caso no município de Maringá, compreendendo o plano original (visto na dissertação), as áreas de expansão urbana da cidade e o território circunstante parcelado pela Companhia de Terras Norte do Paraná (CTNP) – constituído por uma leitura morfológica associada à revisão bibliográfica dos princípios norteadores desse processo de planejamento e projeto.
 
    A opção por continuar com Maringá nesta tese deve-se ao fato de seu planejamento excepcional, posto que fora do usual. Como levantei durante a dissertação, o anteprojeto de Maringá (anteprojeto que é o exemplo de produto resultante do método que pretendo desenvolver) é reputado como uma solução moderna e ao mesmo tempo adequada às pré-existências do sítio – concomitantemente com o parcelamento rural efetuado pela CTNP, que encomendou o traçado de Maringá ao engenheiro Jorge de Macedo Vieira, 'discípulo' do ideário \textit{Garden City}. Além disso, em Maringá existe uma situação análoga ao que se pode observar país afora: uma cidade média com áreas de expansão de 'qualidade' inferior, feitas a partir do \textit{'modus faciendi} atual'.

    Para a segunda fase, de 'sugestão' (de qual artefato deve ser desenvolvido), devo fazer notar que (no contexto da DSR) a 'solução ideal' seria um \textit{software} baseado em um algoritmo que desenhasse traçados urbanos adequados ao sítio, conforme o conceito de \textit{rendimento urbano}, apenas a partir do \textit{input} das curvas de nível – ou mesmo da simples seleção de uma área. No entanto, como isso demandaria uma equipe completa de engenheiros de software, programadores e profissionais afins, opto aqui por aterme ao essencial: um conjunto de passos (método de projeto) que tanto um projetista pode efetuar quanto, no futuro, um software corretamente programado – ainda que com resultados diferentes. E isso porque é esse método de projeto que estaria na base de um software que me serve como 'solução satisfatória', e é a ele que me atenho, junto com a teoria que lhe deve servir de base.  

    Na terceira fase, farei o 'desenvolvimento' do artefato, por meio do processo de projeto de traçados hipotéticos: (1) em áreas já consolidadas de Maringá, (2) em áreas rurais de Maringá previamente parceladas pela CTNP e (3) em áreas de franja peri-urbana, onde  há pressão para o loteamento das parcelas rurais contíguas à mancha urbana existente. E isso considerando (a) apenas as pré-existências naturais (como sobre \textit{tabula rasa}) e (b) considerando pré-existências naturais e antrópicas (como parcelamentos e estradas rurais, e diretrizes viárias). Em todo esse processo, irei verificar quais as melhores soluções de projeto (as mais viáveis durante meu processo de pesquisa), mantendo documentadas as motivações de cada decisão e suas consequências. Tudo isso gerando distintos traçados urbanos como produtos do artefato em desenvolvimento.

    Consolidado o desenvolvimento do método de projeto, com todas as suas etapas definidas, devo ater-me à sua avaliação, utilizando parâmetros determinados na fase de compreensão do problema. Essa avaliação dar-se-á: (1) pela comparação dos traçados hipotéticos com a situação atual (percentuais de viabilidade, legislação) e com os parâmetros de \textit{rendimento urbano}/'qualidade' (que deverão ser melhor desenvolvidos) – procedimento que, por si só, já pode fechar o ciclo de avaliação, posto que "é possível o desenvolvimento de artefatos sem o envolvimento do usuário ou, até mesmo, concebendo artefatos para um usuário 'imaginado'" (Santos, 2018, p. 74); e, em tempo, (2) pelo \textit{feedback} de profissionais, professores e alunos que serão convidados a aplicar o método de projeto em outros contextos por meio de grupos focais (com charretes e outras atividades).\footnote[2]{A importância da participação do usuário final não pode ser menosprezada, seja ele o profissional que desenha o loteamento (e eu diria até quem o encomenda), o gestor público que o aprova junto com as diretrizes viárias, e os alunos que precisam aprender como projetar antes de entrar na vida profissional. Afinal, segundo Stefan Moritz (2005, p. 34), nenhum \textit{design} faz sentido se não preencher a uma necessidade do usuário, por mais bonito e engenhoso que seja. Assim, ainda que não seja imprescindível no âmbito de uma DSR, é importante envolver o usuário no \textit{design}, de modo que este possa, efetivamente, ser útil ao usuário, e, com isso, aplicado no mundo real.} 

    Por fim, na conclusão, o método de projeto será refinado com base no \textit{feedback} recebido na fase anterior. E, eventualmente, será comunicado a pesquisadores, profissionais e gestores na forma de \textit{guidelines} para o projeto de traçados urbanos, de modo a facilitar sua compreensão e aplicabilidade por tais \textit{stakeholders}. 
    
    Desse modo, a tese será estruturada em três capítulos: o primeiro, compreendendo as duas primeiras fases. O segundo com a terceira fase. E o último capítulo contendo as duas últimas fases da DSR aqui proposta. Assim, almeja-se que tal método de projeto, uma vez desenvolvido com rigor científico e comunicado de maneira clara às partes interessadas, corrobore na construção de um novo paradigma de projeto, baseado na morfogênese e não em abstrações que se deformam na medida em que não se adaptam à realidade.



    %Problemas:
    %    – ausência de conceito (morfoadequabilidade)
    %   – ausência de método (método morfogeneticamente adequado de projeto para traçados urbanos – MEPROTU)
    %   – problemas associados à falta de rendimento.

    
    
    
    %\footnote[2]{Quando falo em 'adequação' e 'coerência', penso-as como sinônimo, ou análogos de 'identificação' em sentido filosófico, em que uma coisa é igual à outra. Um traçado urbano deve ser 'idêntico' ao relevo sobre o qual ele se assenta. Em que medida isso se dará? Apenas na medida em que suas formas se adequarem o máximo possível às formas do relevo, identificando-se com elas.  Exemplo clamorosos disso são cidades medievais e favelas. Como reproduzir seu traçado? Mais ainda, como projetar,  em larga escala, um traçado análogo a esses – e sem prejuízo para a implementação de infraestruturas e o acesso do Estado a essas áreas? São pontos importantes a considerar e que serão tratados, primeiro em uma conceituação teórica (que vai além de uma revisão bibliográfica, ou seja, eu vou formar um conceito para poder dar suporte a um método) e depois a verificação de como é possível projetar o que me perguntei.} 
    
   % assim como os assentamentos de origem espontânea. Aparentemente, isso é possível ao se utilizar os parâmetros associados ao conceito de  

%O que seria o "ambiente externo" de uma design science research?
        %ChatGPT
            %O "ambiente externo" em uma design science research refere-se ao contexto ou ao conjunto de elementos e influências externas que podem impactar o processo de pesquisa e o desenvolvimento dos artefatos de design. Nesse contexto, a design science research é conduzida em um ambiente mais amplo que pode incluir fatores externos, como:

                %1. Relevância do problema: O ambiente externo pode incluir a relevância e a importância do problema de pesquisa dentro da comunidade acadêmica, da indústria ou da sociedade em geral. A pesquisa deve abordar questões relevantes e significativas que possam levar a contribuições valiosas para o conhecimento e a prática em design.
                
                %2. Influências sociais e culturais: O ambiente externo também pode ser influenciado pelas tendências sociais e culturais em curso. Por exemplo, as mudanças nas preferências do consumidor ou os avanços tecnológicos podem ter um impacto significativo no design de artefatos e soluções.
                
                %3. Legislação e regulamentações: O ambiente externo pode ser afetado por leis, regulamentações e normas relevantes ao domínio de pesquisa. Os projetos de design devem estar em conformidade com as normas legais e éticas aplicáveis
                
                %4. Tecnologia disponível: As tecnologias disponíveis no momento da pesquisa podem influenciar o escopo e a viabilidade das soluções de design propostas. Por exemplo, a disponibilidade de ferramentas de software específicas pode afetar a forma como um artefato é projetado e implementado.
                
                %5. Desenvolvimentos de pesquisa relacionados: O ambiente externo também pode ser moldado por outras pesquisas e projetos relacionados ao tema em estudo. É importante estar ciente dos trabalhos anteriores e atuais que possam ser relevantes para a pesquisa em andamento.
                
                %6. Restrições e recursos: Fatores financeiros, restrições de tempo e recursos disponíveis podem influenciar o escopo e o planejamento da design science research. Essas limitações devem ser levadas em consideração durante o processo de pesquisa e desenvolvimento dos artefatos.
                
                %7. Parcerias e colaborações: O ambiente externo também pode ser afetado pelas parcerias e colaborações com outras instituições acadêmicas, organizações ou empresas. Essas colaborações podem fornecer acesso a recursos adicionais e expertise complementar.

            % Em resumo, o ambiente externo em uma design science research é composto por todos os fatores e influências que estão fora do escopo imediato do projeto de pesquisa, mas que podem impactar o seu desenvolvimento, relevância e resultados finais. É importante para os pesquisadores considerarem esses elementos externos ao planejar e conduzir suas investigações em design science.

% E o que seria o "ambiente interno"?
    %ChatGPT
        %O "ambiente interno" em uma design science research refere-se aos elementos e características que são inerentes ao próprio processo de pesquisa e desenvolvimento dos artefatos de design. Esse ambiente interno é controlado e influenciado diretamente pelos pesquisadores e pela metodologia adotada na pesquisa. Ele inclui uma série de componentes relacionados ao próprio processo de investigação, tais como:

            %1. Metodologia de pesquisa: O ambiente interno é moldado pela metodologia de pesquisa adotada pelos pesquisadores. Isso inclui a abordagem geral de pesquisa, os métodos de coleta de dados, os procedimentos de análise e avaliação dos resultados. A escolha da metodologia afeta a qualidade e a validade dos artefatos de design criados.

            %2. Hipóteses e objetivos: O ambiente interno é influenciado pelas hipóteses e objetivos estabelecidos para a pesquisa. As hipóteses orientam as investigações e os experimentos realizados para desenvolver e testar os artefatos de design.

            %3. Modelos e teorias: O ambiente interno pode ser enriquecido pela aplicação de modelos e teorias relevantes ao domínio de pesquisa. Esses modelos e teorias fornecem estruturas conceituais para entender o problema e desenvolver soluções.

            %4. Criatividade e inovação: O ambiente interno pode estimular a criatividade e a inovação, pois os pesquisadores buscam encontrar soluções originais e efetivas para os problemas identificados.

            %5. Prototipagem e experimentação: A prototipagem e a experimentação são elementos fundamentais do ambiente interno, pois permitem que os pesquisadores testem e iterem diferentes soluções antes de chegar ao artefato final de design.

            %6. Equipe de pesquisa: A equipe de pesquisa e a colaboração entre os membros desempenham um papel importante no ambiente interno. A interação entre diferentes especialistas pode levar a abordagens multidisciplinares e soluções mais abrangentes.

            %7. Recursos internos: O ambiente interno é influenciado pelos recursos disponíveis para a pesquisa, incluindo financiamento, acesso a equipamentos, software e infraestrutura.

            %8. Avaliação e validação: A avaliação e validação dos artefatos de design são partes essenciais do ambiente interno. Os pesquisadores precisam garantir que os artefatos desenvolvidos atendam aos critérios de sucesso estabelecidos.

        %Em resumo, o ambiente interno em uma design science research é composto pelos elementos controlados e influenciados pelos próprios pesquisadores durante o processo de investigação e desenvolvimento dos artefatos de design. É o conjunto de fatores internos que contribuem para a eficácia, a qualidade e a inovação da pesquisa em design science.

\end{document}